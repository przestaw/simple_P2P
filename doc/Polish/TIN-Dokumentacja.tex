% !TEX TS-program = pdflatex
% !TEX encoding = UTF-8 Unicode

\documentclass[11pt,oneside]{book}
\usepackage[utf8]{inputenc} 
\usepackage[parfill]{parskip}
\usepackage[T1]{fontenc} 

\usepackage{fixltx2e}
\usepackage{calc}
\usepackage{doxygen}
\usepackage[export]{adjustbox} % also loads graphicx
\usepackage{makeidx}
\usepackage{multicol}
\usepackage{multirow}
\PassOptionsToPackage{warn}{textcomp}
\usepackage{textcomp}
\usepackage[nointegrals]{wasysym}
\usepackage[table]{xcolor}

% Font selection
\usepackage[T1]{fontenc}
\usepackage[scaled=.90]{helvet}
\usepackage{courier}
\usepackage{amssymb}
\usepackage{sectsty}
\renewcommand{\familydefault}{\sfdefault}
\allsectionsfont{%
	\fontseries{bc}\selectfont%
	\color{darkgray}%
}
\renewcommand{\DoxyLabelFont}{%
	\fontseries{bc}\selectfont%
	\color{darkgray}%
}
\newcommand{\+}{\discretionary{\mbox{\scriptsize$\hookleftarrow$}}{}{}}

%%% PAGE DIMENSIONS
\usepackage{geometry} % to change the page dimensions
\geometry{a4paper} % or letterpaper (US) or a5paper or....
\geometry{margin=1in} % for example, change the margins to 2 inches all round

\usepackage{graphicx} % support the \includegraphics command and options
\usepackage[parfill]{parskip} % Activate to begin paragraphs with an empty line rather than an indent

%%% PACKAGES
\usepackage{booktabs} % for much better looking tables
\usepackage{array} % for better arrays (eg matrices) in maths
\usepackage{paralist} % very flexible & customisable lists (eg. enumerate/itemize, etc.)
\usepackage{verbatim} % adds environment for commenting out blocks of text & for better verbatim
\usepackage{subfig} % make it possible to include more than one captioned figure/table in a single float
% These packages are all incorporated in the memoir class to one degree or another...
\usepackage{graphicx} 
\graphicspath{ {./images/} }

\usepackage{ifpdf}
\ifpdf
\usepackage[pdftex,pagebackref=true]{hyperref}
\else
\usepackage[ps2pdf,pagebackref=true]{hyperref}
\fi
\hypersetup{%
	colorlinks=true,%
	linkcolor=blue,%
	citecolor=blue,%
	unicode%
}

%%% HEADERS & FOOTERS
\usepackage{fancyhdr} 
\pagestyle{fancy} 
\renewcommand{\headrulewidth}{0pt} 
\lhead{}\chead{}\rhead{}
\lfoot{}\cfoot{\thepage}\rfoot{}
%%% END Article customizations

%%% The "real" document content comes below...

\title{Projekt TIN - Dokumentacja}
\author{Wiktor Michalski\\
Przemysław Stawczyk\\
Maciej Szulik\\
Kamil Zacharczuk
}
% \date{12 Listopada 2019} % Activate to display a given date or no date (if empty),
         % otherwise the current date is printed 

\begin{document}
	\hypersetup{pageanchor=false,
		bookmarksnumbered=true,
		pdfencoding=unicode
	}
	\maketitle
	\setcounter{secnumdepth}{3}
	\setcounter{tocdepth}{3}
	\tableofcontents
	\clearpage
\part{Dokumentacja Koncepcji}
\chapter{Zadanie}
\section{Treść Zadania}
\paragraph{Napisać program obsługujący prosty protokół P2P}
\begin{enumerate}
\item
Zasób to plik identyfikowany pewną nazwą, za takie same zasoby uważa się zasoby o takich samych nazwach i takiej samej wielkości pliku w bajtach.
\item
Początkowo dany zasób znajduje się w jednym węźle sieci, następnie może być propagowany do innych węzłów w ramach inicjowanego przez użytkownika ręcznie transferu - raz pobrany zasób zostaje zachowany jako kopia.
\item
Po pewnym czasie działania systemu ten sam zasób może się znajdować w kilku węzłach sieci (na kilku maszynach).
\item
System ma informować o posiadanych lokalnie (tj. w danym węźle) zasobach i umożliwiać ich pobranie.
\item
Program powinien umożliwiać współbieżne:
\begin{itemize}
\item
wprowadzanie przez użytkownika (poprzez interfejs tekstowy):
\begin{itemize}
\item
nowych zasobów - z lokalnego systemu plików
\item
poleceń pobrania nazwanego zasobu ze zdalnego węzła
\end{itemize}
\item
pobieranie zasobów (także kilku jednocześnie)
\item
rozgłaszanie informacji o posiadanych lokalne zasobach
\end{itemize}
\item
W przypadku pobierania zdalnego zasobu system sam (nie użytkownik) decyduje skąd zostanie on pobrany.
\item
Powinno być możliwe pobranie zasobu z kilku węzłów na raz (tj. "w kawałkach").
\item
Zasób pobrany do lokalnego węzła jest kopią oryginału, kopia jest traktowana tak samo jak oryginał (są nierozróżnialne). Istnienie kopii jest rozgłaszane tak samo jak oryginału.
\item
Właściciel zasobu może go unieważnić wysyłając odpowiedni komunikat rozgłaszany. Wszystkie kopie zasobu powinny przestać być rozgłaszane. W przypadku trwających transferów zasobów powinny się one poprawnie zakończyć, dopiero wtedy informacja o zasobie może zostać usunięta.
\end{enumerate}
\section{Wariant zadania dla zespołu}
4. Opóźnienia dla wybranego węzła - węzeł reaguje, ale (czasami) z dużym opóźnieniem.
\section{Interpretacja Zadania}
\paragraph{Doprecyzowanie treści i dodatkowe założenia}
\begin{itemize}
\item
W związku z tym, że kopia i oryginał są nierozróżnialne, zasób może być unieważniony przez dowolnego użytkownika, który go posiada.
\item
Każdy węzeł okresowo rozgłasza informację o posiadanych zasobach. Unieważnienie pliku oznacza, że żaden z węzłów nie będzie już rozgłaszał faktu posiadania tego pliku.
\item
Unieważnienie wysyłane jest asynchronicznie poprzez broadcast UDP.
\item
W przypadku unieważnienia pliku w trakcie trwającego przesyłu tego pliku przesyłanie kończy się sukcesem, o ile nie wystąpią inne błędy. Nowy posiadacz pliku nie będzie jednak nigdy rozgłaszał o nim informacji.
\item
Każdy węzeł przechowuje listy dostępnych zasobów każdego innego węzła. Po odebraniu rozgłoszenia listy zasobów od innego węzła lista ta jest nadpisywana w pamięci węzła odbierającego. Informacje o węźle, w tym lista jego zasobów, są usuwane w przypadku braku, przez ustalony czas, nadchodzącego rozgłoszenia jego listy zasobów. 
\item
W przypadku połączenia z innym węzłem w celu pobrania od niego pliku oczekiwanie na odpowiedź tego węzła ma pewien timeout. Ponadto, jeżeli węzeł przekracza pewien ustalony czas odpowiedzi (nawet jeżeli nie dochodzi do timeout’u), to inkrementujemy licznik “opóźnień” tego węzła (każdy węzeł przechowuje takie liczniki dla każdego innego węzła). Po osiągnięciu ustalonej wartości licznik ten jest zerowany, a węzeł zliczający nie będzie próbował łączyć się z “opóźnionym” węzłem przez pewien określony czas.
\item
Jeśli węzeł niespodziewanie zakończy połączenie TCP i przestanie rozgłaszać swoją tablicę, to po pewnym czasie pozostałe węzły uznają to za opuszczenie przez niego sieci.
\item
W przypadku gdy pojawi się błąd w trakcie transferu TCP, usuwamy pobrane dane (segmenty) i kończymy wątek pobierający. Ponowne pobieranie od tego węzła będzie odbywać się po ponownym połączeniu z węzłem.
\end{itemize}
\chapter{Opis Funkcjonalny Projektu}
\paragraph{Użytkownik systemu ma wgląd w dwie listy}
\begin{itemize}
\item
lokalny rejestr zasobów - pliki, które użytkownik dodał lub pobrał od innych,
\item
pliki obecne w systemie - pliki posiadane w lokalnym rejestrze zasobów przez innych użytkowników, które nie zostały unieważnione.
\end{itemize}
\textsl{Dla każdego użytkownika generowana jest, oczywiście, odrębna para list.}

\paragraph{Użytkownik może wprowadzać tekstowe komendy, aby za ich pomocą}
\begin{itemize}
\item
\textsl{wyświetlić listę lokalnych zasobów,}
\item 
\textsl{wyświetlić listę zasobów obecnych w systemie,}
\item
\textsl{wyświetlić listę dostępnych komend,}
\item
\textsl{opuścić system,}
\end{itemize}
a także wykonywać operacje na plikach, wśród których rozróżniamy:
\begin{itemize}
\item
\textsl{dodanie pliku,}\\
można dodać do zasobów plik, którego nazwa nie wystąpiła jeszcze wśród plików w lokalnym rejestrze zasobów i reszcie sieci.\\
Zakładamy, że nie wystąpi sytuacja, gdy więcej niż jeden użytkownik doda plik o tej samej nazwie “jednocześnie” - to znaczy przed “zauważeniem” przez całą sieć dodania pliku o takiej nazwie przez któregokolwiek z nich.
\item
\textsl{usunięcie pliku,}\\
można usunąć plik z własnego rejestru zasobów.
\item
\textsl{unieważnienie pliku,}\\
można unieważnić plik, który mamy we własnym rejestrze zasobów. Oznacza to, że zasób nie będzie już widoczny na liście plików dostępnych w systemie, ale dotychczasowi posiadacze nadal będą go posiadali w swoim lokalnym systemie plików.
\item
\textsl{pobranie pliku,}\\
można pobrać plik, którego nie mamy jeszcze w rejestrze zasobów, a który jest obecny w systemie.
\end{itemize}
\chapter{Protokół}
\section{Składnia :}
\begin{verbatim}
<Command 1 octet> ::= <REVOKE> | <FILE_LIST> | <REQ_CONN> | <REQ_SEGMENT> 

<ResourceHeader> ::= 
        <nazwa pliku: 256 octets, NULL terminated> <rozmiar pliku: 64bit>

<Message> ::= <Command> | <Command><ResourceHeader> | 
        <Command><ResourceHeader><Options> | 
        <Command><No_Of_Files 16bit><ResourceHeader><ResourceHeader>...

<Resource> ::= <1 KB of file>
\end{verbatim}
\section{Komendy :}
\begin{itemize}
\item
\textsl{Unieważnienie pliku:}\\
Broadcast po UDP: \\ \verb|<Command = REVOKE><ResourceHeader = Revoked File>|
\item
\textsl{Rozgłaszanie dostępnych plików:}\\
Broadcast po UDP: \\ \verb|<Command = FILE_LIST><No_Of_Files = liczba dostępnych plików>| \\
\verb|        <ResourceHeader = plik1>....|
%\item
%\textsl{Wyjście z sieci:}\\
%Broadcast po UDP: \\ \verb|<Command = EXIT_NETWORK>|
\item
\textsl{Żądanie utworzenia połączenia TCP:}\\
Wysyłane do węzła po TCP: \verb|<Command = REQ_CONN>|
\item
\textsl{Żądanie pobrania segmentu:}\\
Wysyłane do węzła po TCP: \\ \verb|<Command = REQ_SEGMENT><ResourceHeader = plik><Options = segment number>|
%\item
%\textsl{Jawne zakończenie pobierania od węzła:}\\
%Wysyłane do węzła po TCP: \\ \verb|<Command = QUIT_CONN>|
%\item
%\textsl{Opuszczenie sieci:}\\
%Broadcast na UDP: \\ \verb|<Command = EXIT_NETWORK>|
\end{itemize}
\chapter{Organizacja Projektu}
\section{Moduły}
\begin{enumerate}
\item
Moduł CLI odpowiedzialny za komunikację z użytkownikiem.
\item
Moduł obsługi sieci.
\item
Moduł dispatchera obsługujący protokół.
\end{enumerate}
\section{Współbieżność}
\paragraph{Ogólna koncepcja zakłada istnienie następujących bazowych, działających w pętli wątków:}
\begin{enumerate}
\item 
\textsl{Obsługa przychodzących żądań przesłania posiadanego zasobu.}\\
Wątek ten nasłuchuje na porcie TCP. W przypadku nawiązania połączenia na tym porcie tworzony jest watek potomny. Wątek ten odbiera żądanie przesłania lokalnie posiadanego pliku i nadzoruje to przesłanie. 
\item
\textsl{Odbiór komunikatów broadcast UDP.}\\
Komunikaty te obejmują okresowe rozgłaszanie przez każdego użytkownika listy lokalnie posiadanych zasobów, rozgłaszanie unieważnienia zasobu. Po odebraniu komunikatu wątek ten przekazuje otrzymane informacje do wątku synchronizującego dane.
\item
\textsl{Okresowe rozgłaszanie lokalnej listy zasobów.}\\
Wątek, który co pewien czas rozgłasza przez UDP listę zasobów, które udostępnia do pobrania. Wątek blokuje się pomiędzy kolejnymi broadcastami.  
\item
\textsl{Wątek synchronizujący dane.}\\
Wątek manipulujący danymi przechowywanymi przez program takimi jak: lista  lokalnych zasobów, informacje o pozostałych węzłach - dla każdego z nich lista zasobów, które udostępnia, liczniki opóźnień itp.
\item
\textsl{Obsługa interfejsu użytkownika.}\\
Interakcja z użytkownikiem przez CLI. Odbieranie komend od użytkownika i odpowiednie reagowanie - powoływanie nowych wątków, które mają zająć się realizacją komendy, między innymi: 
\begin{itemize}
\item
W przypadku chęci pobrania pliku tworzony jest wątek nadzorujący to pobieranie. Na potrzeby połączenia z wieloma węzłami może on stworzyć kilka kolejnych wątków przypisanych do połączeń z węzłami.\\
W zależności od wyniku pobierania przekaże odpowiednie informacje do wątku synchronizującego dane.
\item
W przypadku chęci dodania, usunięcia lub unieważnienia pliku tworzony jest wątek, który zajmie się wprowadzeniem tej zmiany i nadzorowaniem wszystkich jej następstw, np.: fizycznie doda plik do systemu i przekaże informację o nowym pliku do wątku synchronizującego dane. W przypadku unieważnienia stworzy nowy wątek, który rozgłosi odpowiednią informację w systemie.
\item
W przypadku chęci wyświetlenia którejś z list zasobów lub listy dostępnych komend tworzony jest wątek, który odczyta odpowiednie dane i przygotuje je w odpowiedniej formie do wyświetlenia użytkownikowi.
\end{itemize}
\textbf{Wyniki działań powyższych wątków przekazywane są z powrotem do wątku obsługującego CLI, który wyświetla je użytkownikowi.}
\end{enumerate}
\chapter{Implementacja}
\begin{itemize}
\item
Jezyki : \textsl{C++}
\item
Biblioteki : \textsl{Boost:Asio, std::thread}
\end{itemize}

%% BEGIN DOXYGEN
\part{Dokumentacja kodu źródłowego - \textit{English}}
\label{md_README}
\Hypertarget{md_README}
Description\+: T\+O\+DO 

 Build\+: Project Uses C\+Make based config. \textquotesingle{}create\+\_\+configs.\+sh\textquotesingle{} shell script will generate ninja files for compilation. Boost and thearding libaries are required 
\chapter{Class Index}
\section{Class List}
Here are the classes, structs, unions and interfaces with brief descriptions\+:\begin{DoxyCompactList}
\item\contentsline{section}{\hyperlink{classSimpleP2P_1_1FileManager}{Simple\+P2\+P\+::\+File\+Manager} \\*Handles read/write to the files on disc }{\pageref{classSimpleP2P_1_1FileManager}}{}
\item\contentsline{section}{\hyperlink{classSimpleP2P_1_1FileRequest}{Simple\+P2\+P\+::\+File\+Request} \\*Carries info about a single file transfer request -\/ resource header and numbers of wanted segments }{\pageref{classSimpleP2P_1_1FileRequest}}{}
\item\contentsline{section}{\hyperlink{classsimpleP2P_1_1Host}{simple\+P2\+P\+::\+Host} \\*Forward declaration }{\pageref{classsimpleP2P_1_1Host}}{}
\item\contentsline{section}{\hyperlink{classsimpleP2P_1_1Logging__Module}{simple\+P2\+P\+::\+Logging\+\_\+\+Module} \\*Class Providing logging support based on text logs }{\pageref{classsimpleP2P_1_1Logging__Module}}{}
\item\contentsline{section}{\hyperlink{classSimpleP2P_1_1RequestServer}{Simple\+P2\+P\+::\+Request\+Server} \\*Asynchronous T\+CP server }{\pageref{classSimpleP2P_1_1RequestServer}}{}
\item\contentsline{section}{\hyperlink{classSimpleP2P_1_1RequestServerModule}{Simple\+P2\+P\+::\+Request\+Server\+Module} \\*Module of the T\+CP server receiving file requests and sending the requested files\textquotesingle{} segments }{\pageref{classSimpleP2P_1_1RequestServerModule}}{}
\item\contentsline{section}{\hyperlink{classSimpleP2P_1_1RequestWorker}{Simple\+P2\+P\+::\+Request\+Worker} \\*T\+CP connection handler, created by the T\+CP server }{\pageref{classSimpleP2P_1_1RequestWorker}}{}
\item\contentsline{section}{\hyperlink{classsimpleP2P_1_1Resource}{simple\+P2\+P\+::\+Resource} \\*Forward declaration }{\pageref{classsimpleP2P_1_1Resource}}{}
\item\contentsline{section}{\hyperlink{classsimpleP2P_1_1Resource__Database}{simple\+P2\+P\+::\+Resource\+\_\+\+Database} \\*Class holding information about files in network and on localhost }{\pageref{classsimpleP2P_1_1Resource__Database}}{}
\item\contentsline{section}{\hyperlink{classsimpleP2P_1_1Udp__Client}{simple\+P2\+P\+::\+Udp\+\_\+\+Client} \\*Class U\+DP Client to handle all outgoing packets }{\pageref{classsimpleP2P_1_1Udp__Client}}{}
\item\contentsline{section}{\hyperlink{classsimpleP2P_1_1Udp__Module}{simple\+P2\+P\+::\+Udp\+\_\+\+Module} \\*Class containing all U\+DP related resources and logic }{\pageref{classsimpleP2P_1_1Udp__Module}}{}
\item\contentsline{section}{\hyperlink{classsimpleP2P_1_1Udp__Server}{simple\+P2\+P\+::\+Udp\+\_\+\+Server} \\*Class U\+DP Server to handle all incoming packets }{\pageref{classsimpleP2P_1_1Udp__Server}}{}
\end{DoxyCompactList}

\chapter{Class Documentation}
\hypertarget{classsimpleP2P_1_1Logging__Module}{}\subsection{simple\+P2P\+:\+:Logging\+\_\+\+Module Class Reference}
\label{classsimpleP2P_1_1Logging__Module}\index{simple\+P2\+P\+::\+Logging\+\_\+\+Module@{simple\+P2\+P\+::\+Logging\+\_\+\+Module}}


class Providing logging support based on text logs  




{\ttfamily \#include $<$logging\+\_\+module.\+h$>$}

\subsubsection*{Public Member Functions}
\begin{DoxyCompactItemize}
\item 
\hyperlink{classsimpleP2P_1_1Logging__Module_a300ea6b6f0c06caa7891c2226fecf67b}{Logging\+\_\+\+Module} (std\+::ostream \&output\+\_\+c=std\+::cerr)
\begin{DoxyCompactList}\small\item\em Constructor for the logging thread. \end{DoxyCompactList}\item 
std\+::thread \hyperlink{classsimpleP2P_1_1Logging__Module_a57e92e151320fd7c811689d831498b6a}{init} ()
\begin{DoxyCompactList}\small\item\em Init methods run worker in thread and returns it. \end{DoxyCompactList}\item 
void \hyperlink{classsimpleP2P_1_1Logging__Module_a5f1eb6d6c66d406f02ae4093cc7657ec}{add\+\_\+log\+\_\+line} (std\+::string line, const std\+::time\+\_\+t time)
\begin{DoxyCompactList}\small\item\em Synchronised method for logging output. \end{DoxyCompactList}\end{DoxyCompactItemize}


\subsubsection{Detailed Description}
class Providing logging support based on text logs 

\subsubsection{Constructor \& Destructor Documentation}
\mbox{\Hypertarget{classsimpleP2P_1_1Logging__Module_a300ea6b6f0c06caa7891c2226fecf67b}\label{classsimpleP2P_1_1Logging__Module_a300ea6b6f0c06caa7891c2226fecf67b}} 
\index{simple\+P2\+P\+::\+Logging\+\_\+\+Module@{simple\+P2\+P\+::\+Logging\+\_\+\+Module}!Logging\+\_\+\+Module@{Logging\+\_\+\+Module}}
\index{Logging\+\_\+\+Module@{Logging\+\_\+\+Module}!simple\+P2\+P\+::\+Logging\+\_\+\+Module@{simple\+P2\+P\+::\+Logging\+\_\+\+Module}}
\paragraph{\texorpdfstring{Logging\+\_\+\+Module()}{Logging\_Module()}}
{\footnotesize\ttfamily simple\+P2\+P\+::\+Logging\+\_\+\+Module\+::\+Logging\+\_\+\+Module (\begin{DoxyParamCaption}\item[{std\+::ostream \&}]{output\+\_\+c = {\ttfamily std\+:\+:cerr} }\end{DoxyParamCaption})}



Constructor for the logging thread. 


\begin{DoxyParams}{Parameters}
{\em output\+\_\+c} & Output stream for the logs \\
\hline
\end{DoxyParams}
\begin{DoxyNote}{Note}
if output stream is a file you must explicitly close it 
\end{DoxyNote}


\subsubsection{Member Function Documentation}
\mbox{\Hypertarget{classsimpleP2P_1_1Logging__Module_a5f1eb6d6c66d406f02ae4093cc7657ec}\label{classsimpleP2P_1_1Logging__Module_a5f1eb6d6c66d406f02ae4093cc7657ec}} 
\index{simple\+P2\+P\+::\+Logging\+\_\+\+Module@{simple\+P2\+P\+::\+Logging\+\_\+\+Module}!add\+\_\+log\+\_\+line@{add\+\_\+log\+\_\+line}}
\index{add\+\_\+log\+\_\+line@{add\+\_\+log\+\_\+line}!simple\+P2\+P\+::\+Logging\+\_\+\+Module@{simple\+P2\+P\+::\+Logging\+\_\+\+Module}}
\paragraph{\texorpdfstring{add\+\_\+log\+\_\+line()}{add\_log\_line()}}
{\footnotesize\ttfamily void simple\+P2\+P\+::\+Logging\+\_\+\+Module\+::add\+\_\+log\+\_\+line (\begin{DoxyParamCaption}\item[{std\+::string}]{line,  }\item[{const std\+::time\+\_\+t}]{time }\end{DoxyParamCaption})}



Synchronised method for logging output. 


\begin{DoxyParams}{Parameters}
{\em line} & \\
\hline
\end{DoxyParams}
\mbox{\Hypertarget{classsimpleP2P_1_1Logging__Module_a57e92e151320fd7c811689d831498b6a}\label{classsimpleP2P_1_1Logging__Module_a57e92e151320fd7c811689d831498b6a}} 
\index{simple\+P2\+P\+::\+Logging\+\_\+\+Module@{simple\+P2\+P\+::\+Logging\+\_\+\+Module}!init@{init}}
\index{init@{init}!simple\+P2\+P\+::\+Logging\+\_\+\+Module@{simple\+P2\+P\+::\+Logging\+\_\+\+Module}}
\paragraph{\texorpdfstring{init()}{init()}}
{\footnotesize\ttfamily std\+::thread simple\+P2\+P\+::\+Logging\+\_\+\+Module\+::init (\begin{DoxyParamCaption}{ }\end{DoxyParamCaption})}



Init methods run worker in thread and returns it. 

\begin{DoxyReturn}{Returns}
logging thread 
\end{DoxyReturn}


The documentation for this class was generated from the following files\+:\begin{DoxyCompactItemize}
\item 
include/logging\+\_\+module.\+h\item 
src/logging\+\_\+module.\+cpp\end{DoxyCompactItemize}

\hypertarget{classSimpleP2P_1_1RequestWorker}{}\subsection{Simple\+P2P\+:\+:Request\+Worker Class Reference}
\label{classSimpleP2P_1_1RequestWorker}\index{Simple\+P2\+P\+::\+Request\+Worker@{Simple\+P2\+P\+::\+Request\+Worker}}


T\+CP connection handler, created by the T\+CP server.  




{\ttfamily \#include $<$Request\+Worker.\+h$>$}



Inherits enable\+\_\+shared\+\_\+from\+\_\+this$<$ Request\+Worker $>$.

\subsubsection*{Public Member Functions}
\begin{DoxyCompactItemize}
\item 
\mbox{\Hypertarget{classSimpleP2P_1_1RequestWorker_abccdef4b69527693cc3d8681cead05b1}\label{classSimpleP2P_1_1RequestWorker_abccdef4b69527693cc3d8681cead05b1}} 
\hyperlink{classSimpleP2P_1_1RequestWorker_abccdef4b69527693cc3d8681cead05b1}{Request\+Worker} (boost\+::asio\+::io\+\_\+service \&io\+\_\+service)
\begin{DoxyCompactList}\small\item\em Constructor allows setting the socket on which the connection is established. \end{DoxyCompactList}\item 
\mbox{\Hypertarget{classSimpleP2P_1_1RequestWorker_a08b2e9500f0e6d84c0d79f82207dd16d}\label{classSimpleP2P_1_1RequestWorker_a08b2e9500f0e6d84c0d79f82207dd16d}} 
void \hyperlink{classSimpleP2P_1_1RequestWorker_a08b2e9500f0e6d84c0d79f82207dd16d}{start} ()
\begin{DoxyCompactList}\small\item\em Start handling the request. \end{DoxyCompactList}\item 
\mbox{\Hypertarget{classSimpleP2P_1_1RequestWorker_abcac032e7b8a594f375bb36f92beed64}\label{classSimpleP2P_1_1RequestWorker_abcac032e7b8a594f375bb36f92beed64}} 
tcp\+::socket \& \hyperlink{classSimpleP2P_1_1RequestWorker_abcac032e7b8a594f375bb36f92beed64}{socket} ()
\begin{DoxyCompactList}\small\item\em Get socket. \end{DoxyCompactList}\end{DoxyCompactItemize}


\subsubsection{Detailed Description}
T\+CP connection handler, created by the T\+CP server. 

Receives the file request, buffers requested segments and sends them to the client. 

The documentation for this class was generated from the following files\+:\begin{DoxyCompactItemize}
\item 
include/Request\+Worker.\+h\item 
src/Request\+Worker.\+cpp\end{DoxyCompactItemize}

\hypertarget{classSimpleP2P_1_1RequestServer}{}\section{Simple\+P2P\+:\+:Request\+Server Class Reference}
\label{classSimpleP2P_1_1RequestServer}\index{Simple\+P2\+P\+::\+Request\+Server@{Simple\+P2\+P\+::\+Request\+Server}}


Asynchronous T\+CP server.  




{\ttfamily \#include $<$Request\+Server.\+h$>$}

\subsection*{Public Member Functions}
\begin{DoxyCompactItemize}
\item 
\hyperlink{classSimpleP2P_1_1RequestServer_adfd41a4161009a119b523e731a21d85d}{Request\+Server} (boost\+::asio\+::io\+\_\+service \&io\+\_\+service, Uint16 port)
\begin{DoxyCompactList}\small\item\em Constructor allows setting the parameters for the connnetion acceptor. \end{DoxyCompactList}\item 
\mbox{\Hypertarget{classSimpleP2P_1_1RequestServer_afe2b69eb9717c03f083a6060f83570d3}\label{classSimpleP2P_1_1RequestServer_afe2b69eb9717c03f083a6060f83570d3}} 
std\+::thread \hyperlink{classSimpleP2P_1_1RequestServer_afe2b69eb9717c03f083a6060f83570d3}{init} ()
\begin{DoxyCompactList}\small\item\em Turns on the listening and accepting connections and returns the thread in which it works. \end{DoxyCompactList}\end{DoxyCompactItemize}


\subsection{Detailed Description}
Asynchronous T\+CP server. 

It accepts connections asynchronously and for each of them creates a worker object to handle it. 

\subsection{Constructor \& Destructor Documentation}
\mbox{\Hypertarget{classSimpleP2P_1_1RequestServer_adfd41a4161009a119b523e731a21d85d}\label{classSimpleP2P_1_1RequestServer_adfd41a4161009a119b523e731a21d85d}} 
\index{Simple\+P2\+P\+::\+Request\+Server@{Simple\+P2\+P\+::\+Request\+Server}!Request\+Server@{Request\+Server}}
\index{Request\+Server@{Request\+Server}!Simple\+P2\+P\+::\+Request\+Server@{Simple\+P2\+P\+::\+Request\+Server}}
\subsubsection{\texorpdfstring{Request\+Server()}{RequestServer()}}
{\footnotesize\ttfamily Simple\+P2\+P\+::\+Request\+Server\+::\+Request\+Server (\begin{DoxyParamCaption}\item[{boost\+::asio\+::io\+\_\+service \&}]{io\+\_\+service,  }\item[{Uint16}]{port }\end{DoxyParamCaption})}



Constructor allows setting the parameters for the connnetion acceptor. 


\begin{DoxyParams}{Parameters}
{\em io\+\_\+service} & boost\+::asio\+::io\+\_\+service for the acceptor. \\
\hline
{\em port} & Port for the acceptor to listen on. \\
\hline
\end{DoxyParams}


The documentation for this class was generated from the following files\+:\begin{DoxyCompactItemize}
\item 
include/Request\+Server.\+h\item 
src/Request\+Server.\+cpp\end{DoxyCompactItemize}

\hypertarget{classSimpleP2P_1_1RequestServerModule}{}\subsection{Simple\+P2P\+:\+:Request\+Server\+Module Class Reference}
\label{classSimpleP2P_1_1RequestServerModule}\index{Simple\+P2\+P\+::\+Request\+Server\+Module@{Simple\+P2\+P\+::\+Request\+Server\+Module}}


Module of the T\+CP server receiving file requests and sending the requested files\textquotesingle{} segments.  




{\ttfamily \#include $<$Request\+Server\+Module.\+h$>$}

\subsubsection*{Public Member Functions}
\begin{DoxyCompactItemize}
\item 
\mbox{\Hypertarget{classSimpleP2P_1_1RequestServerModule_af2338209b76cc5bdf26ea16ee7532579}\label{classSimpleP2P_1_1RequestServerModule_af2338209b76cc5bdf26ea16ee7532579}} 
\hyperlink{classSimpleP2P_1_1RequestServerModule_af2338209b76cc5bdf26ea16ee7532579}{Request\+Server\+Module} (Uint16 port\+\_\+)
\begin{DoxyCompactList}\small\item\em Constructor, allows setting the port for the server. \end{DoxyCompactList}\item 
std\+::thread \hyperlink{classSimpleP2P_1_1RequestServerModule_aa8b342854453de8c7050062fdafbf6cd}{init} ()
\begin{DoxyCompactList}\small\item\em Returns the thread object for the module. \end{DoxyCompactList}\end{DoxyCompactItemize}


\subsubsection{Detailed Description}
Module of the T\+CP server receiving file requests and sending the requested files\textquotesingle{} segments. 

\subsubsection{Member Function Documentation}
\mbox{\Hypertarget{classSimpleP2P_1_1RequestServerModule_aa8b342854453de8c7050062fdafbf6cd}\label{classSimpleP2P_1_1RequestServerModule_aa8b342854453de8c7050062fdafbf6cd}} 
\index{Simple\+P2\+P\+::\+Request\+Server\+Module@{Simple\+P2\+P\+::\+Request\+Server\+Module}!init@{init}}
\index{init@{init}!Simple\+P2\+P\+::\+Request\+Server\+Module@{Simple\+P2\+P\+::\+Request\+Server\+Module}}
\paragraph{\texorpdfstring{init()}{init()}}
{\footnotesize\ttfamily std\+::thread Simple\+P2\+P\+::\+Request\+Server\+Module\+::init (\begin{DoxyParamCaption}{ }\end{DoxyParamCaption})}



Returns the thread object for the module. 

Starts the server and returns the thread in which the server works. The thread of the server. 

The documentation for this class was generated from the following files\+:\begin{DoxyCompactItemize}
\item 
include/Request\+Server\+Module.\+h\item 
src/Request\+Server\+Module.\+cpp\end{DoxyCompactItemize}

\hypertarget{classsimpleP2P_1_1Udp__Client}{}\section{simple\+P2P\+:\+:Udp\+\_\+\+Client Class Reference}
\label{classsimpleP2P_1_1Udp__Client}\index{simple\+P2\+P\+::\+Udp\+\_\+\+Client@{simple\+P2\+P\+::\+Udp\+\_\+\+Client}}


class U\+DP Client to handle all outgoing packets  




{\ttfamily \#include $<$udp\+\_\+client.\+h$>$}



Inherits enable\+\_\+shared\+\_\+from\+\_\+this$<$ Udp\+\_\+\+Client $>$.

\subsection*{Public Member Functions}
\begin{DoxyCompactItemize}
\item 
\hyperlink{classsimpleP2P_1_1Udp__Client_acb5ed96ec556665f56510c27f234bb5a}{Udp\+\_\+\+Client} (boost\+::asio\+::io\+\_\+service \&io\+\_\+service, const boost\+::asio\+::ip\+::address \&broadcast\+\_\+address, Uint16 broadcast\+\_\+port, Uint32 timeout=5 $\ast$60)
\begin{DoxyCompactList}\small\item\em Constructor of U\+DP Client. \end{DoxyCompactList}\item 
\mbox{\Hypertarget{classsimpleP2P_1_1Udp__Client_a710c95899ddcfba71004d780ab4c1e4d}\label{classsimpleP2P_1_1Udp__Client_a710c95899ddcfba71004d780ab4c1e4d}} 
\hyperlink{classsimpleP2P_1_1Udp__Client_a710c95899ddcfba71004d780ab4c1e4d}{$\sim$\+Udp\+\_\+\+Client} ()
\begin{DoxyCompactList}\small\item\em Destructor closes socket. \end{DoxyCompactList}\item 
void \hyperlink{classsimpleP2P_1_1Udp__Client_a459f622d0f3671d48d33bce035f5de90}{revoke\+\_\+file} (\hyperlink{classsimpleP2P_1_1Resource}{Resource} resource)
\begin{DoxyCompactList}\small\item\em Constructs revoke header sends it. \end{DoxyCompactList}\end{DoxyCompactItemize}


\subsection{Detailed Description}
class U\+DP Client to handle all outgoing packets 

\subsection{Constructor \& Destructor Documentation}
\mbox{\Hypertarget{classsimpleP2P_1_1Udp__Client_acb5ed96ec556665f56510c27f234bb5a}\label{classsimpleP2P_1_1Udp__Client_acb5ed96ec556665f56510c27f234bb5a}} 
\index{simple\+P2\+P\+::\+Udp\+\_\+\+Client@{simple\+P2\+P\+::\+Udp\+\_\+\+Client}!Udp\+\_\+\+Client@{Udp\+\_\+\+Client}}
\index{Udp\+\_\+\+Client@{Udp\+\_\+\+Client}!simple\+P2\+P\+::\+Udp\+\_\+\+Client@{simple\+P2\+P\+::\+Udp\+\_\+\+Client}}
\subsubsection{\texorpdfstring{Udp\+\_\+\+Client()}{Udp\_Client()}}
{\footnotesize\ttfamily simple\+P2\+P\+::\+Udp\+\_\+\+Client\+::\+Udp\+\_\+\+Client (\begin{DoxyParamCaption}\item[{boost\+::asio\+::io\+\_\+service \&}]{io\+\_\+service,  }\item[{const boost\+::asio\+::ip\+::address \&}]{broadcast\+\_\+address,  }\item[{Uint16}]{broadcast\+\_\+port,  }\item[{Uint32}]{timeout = {\ttfamily 5~$\ast$~60} }\end{DoxyParamCaption})}



Constructor of U\+DP Client. 


\begin{DoxyParams}{Parameters}
{\em io\+\_\+service} & asio Io Service \\
\hline
{\em broadcast\+\_\+address} & address on which packets will be sent \\
\hline
{\em broadcast\+\_\+port} & port on which packets will be sent \\
\hline
{\em timeout} & beacon interval \\
\hline
\end{DoxyParams}


\subsection{Member Function Documentation}
\mbox{\Hypertarget{classsimpleP2P_1_1Udp__Client_a459f622d0f3671d48d33bce035f5de90}\label{classsimpleP2P_1_1Udp__Client_a459f622d0f3671d48d33bce035f5de90}} 
\index{simple\+P2\+P\+::\+Udp\+\_\+\+Client@{simple\+P2\+P\+::\+Udp\+\_\+\+Client}!revoke\+\_\+file@{revoke\+\_\+file}}
\index{revoke\+\_\+file@{revoke\+\_\+file}!simple\+P2\+P\+::\+Udp\+\_\+\+Client@{simple\+P2\+P\+::\+Udp\+\_\+\+Client}}
\subsubsection{\texorpdfstring{revoke\+\_\+file()}{revoke\_file()}}
{\footnotesize\ttfamily void simple\+P2\+P\+::\+Udp\+\_\+\+Client\+::revoke\+\_\+file (\begin{DoxyParamCaption}\item[{\hyperlink{classsimpleP2P_1_1Resource}{simple\+P2\+P\+::\+Resource}}]{resource }\end{DoxyParamCaption})}



Constructs revoke header sends it. 


\begin{DoxyParams}{Parameters}
{\em resource} & \hyperlink{classsimpleP2P_1_1Resource}{Resource} to be revoked \\
\hline
\end{DoxyParams}


The documentation for this class was generated from the following files\+:\begin{DoxyCompactItemize}
\item 
include/udp\+\_\+client.\+h\item 
src/udp\+\_\+client.\+cpp\end{DoxyCompactItemize}

\hypertarget{classsimpleP2P_1_1Udp__Server}{}\subsection{simple\+P2P\+:\+:Udp\+\_\+\+Server Class Reference}
\label{classsimpleP2P_1_1Udp__Server}\index{simple\+P2\+P\+::\+Udp\+\_\+\+Server@{simple\+P2\+P\+::\+Udp\+\_\+\+Server}}


class U\+DP Server to handle all incoming packets  




{\ttfamily \#include $<$udp\+\_\+server.\+h$>$}



Inherits enable\+\_\+shared\+\_\+from\+\_\+this$<$ Udp\+\_\+\+Server $>$.

\subsubsection*{Public Member Functions}
\begin{DoxyCompactItemize}
\item 
\hyperlink{classsimpleP2P_1_1Udp__Server_ad2e7873688dea22fa7db5ae1ef86ba6f}{Udp\+\_\+\+Server} (boost\+::asio\+::io\+\_\+service \&io\+\_\+service, const boost\+::asio\+::ip\+::address \&broadcast\+\_\+address, Uint16 broadcast\+\_\+port)
\begin{DoxyCompactList}\small\item\em Constructor of U\+DP Server. \end{DoxyCompactList}\item 
\mbox{\Hypertarget{classsimpleP2P_1_1Udp__Server_adde5307f34477b9bca29f8e2cb9077b0}\label{classsimpleP2P_1_1Udp__Server_adde5307f34477b9bca29f8e2cb9077b0}} 
\hyperlink{classsimpleP2P_1_1Udp__Server_adde5307f34477b9bca29f8e2cb9077b0}{$\sim$\+Udp\+\_\+\+Server} ()
\begin{DoxyCompactList}\small\item\em Destructor closes socket. \end{DoxyCompactList}\end{DoxyCompactItemize}


\subsubsection{Detailed Description}
class U\+DP Server to handle all incoming packets 

\subsubsection{Constructor \& Destructor Documentation}
\mbox{\Hypertarget{classsimpleP2P_1_1Udp__Server_ad2e7873688dea22fa7db5ae1ef86ba6f}\label{classsimpleP2P_1_1Udp__Server_ad2e7873688dea22fa7db5ae1ef86ba6f}} 
\index{simple\+P2\+P\+::\+Udp\+\_\+\+Server@{simple\+P2\+P\+::\+Udp\+\_\+\+Server}!Udp\+\_\+\+Server@{Udp\+\_\+\+Server}}
\index{Udp\+\_\+\+Server@{Udp\+\_\+\+Server}!simple\+P2\+P\+::\+Udp\+\_\+\+Server@{simple\+P2\+P\+::\+Udp\+\_\+\+Server}}
\paragraph{\texorpdfstring{Udp\+\_\+\+Server()}{Udp\_Server()}}
{\footnotesize\ttfamily simple\+P2\+P\+::\+Udp\+\_\+\+Server\+::\+Udp\+\_\+\+Server (\begin{DoxyParamCaption}\item[{boost\+::asio\+::io\+\_\+service \&}]{io\+\_\+service,  }\item[{const boost\+::asio\+::ip\+::address \&}]{broadcast\+\_\+address,  }\item[{Uint16}]{broadcast\+\_\+port }\end{DoxyParamCaption})}



Constructor of U\+DP Server. 


\begin{DoxyParams}{Parameters}
{\em io\+\_\+service} & asio Io Service \\
\hline
{\em broadcast\+\_\+address} & address on which Server will listen \\
\hline
{\em broadcast\+\_\+port} & port on which Server will listen \\
\hline
\end{DoxyParams}


The documentation for this class was generated from the following files\+:\begin{DoxyCompactItemize}
\item 
include/udp\+\_\+server.\+h\item 
src/udp\+\_\+server.\+cpp\end{DoxyCompactItemize}

\hypertarget{classsimpleP2P_1_1Udp__Module}{}\subsection{simple\+P2P\+:\+:Udp\+\_\+\+Module Class Reference}
\label{classsimpleP2P_1_1Udp__Module}\index{simple\+P2\+P\+::\+Udp\+\_\+\+Module@{simple\+P2\+P\+::\+Udp\+\_\+\+Module}}


Class containing all U\+DP related resources and logic.  




{\ttfamily \#include $<$udp\+\_\+module.\+h$>$}

\subsubsection*{Public Member Functions}
\begin{DoxyCompactItemize}
\item 
\hyperlink{classsimpleP2P_1_1Udp__Module_ae82263553672721e6a6ec034617e08ba}{Udp\+\_\+\+Module} (boost\+::asio\+::ip\+::address broadcast\+\_\+address, Uint16 port, Uint32 beacon\+\_\+interval)
\begin{DoxyCompactList}\small\item\em Constructor. \end{DoxyCompactList}\item 
std\+::thread \hyperlink{classsimpleP2P_1_1Udp__Module_a6e1598a38a557ebdf6eb4849b3cde9db}{init} ()
\begin{DoxyCompactList}\small\item\em Init methods run worker in thread and returns it. \end{DoxyCompactList}\item 
void \hyperlink{classsimpleP2P_1_1Udp__Module_ad3904b2be4e113a8f7be11b23cfde4d2}{revoke\+\_\+file} (const \hyperlink{classsimpleP2P_1_1Resource}{Resource} \&resource)
\begin{DoxyCompactList}\small\item\em Sends revoke datagram. \end{DoxyCompactList}\end{DoxyCompactItemize}


\subsubsection{Detailed Description}
Class containing all U\+DP related resources and logic. 

\subsubsection{Constructor \& Destructor Documentation}
\mbox{\Hypertarget{classsimpleP2P_1_1Udp__Module_ae82263553672721e6a6ec034617e08ba}\label{classsimpleP2P_1_1Udp__Module_ae82263553672721e6a6ec034617e08ba}} 
\index{simple\+P2\+P\+::\+Udp\+\_\+\+Module@{simple\+P2\+P\+::\+Udp\+\_\+\+Module}!Udp\+\_\+\+Module@{Udp\+\_\+\+Module}}
\index{Udp\+\_\+\+Module@{Udp\+\_\+\+Module}!simple\+P2\+P\+::\+Udp\+\_\+\+Module@{simple\+P2\+P\+::\+Udp\+\_\+\+Module}}
\paragraph{\texorpdfstring{Udp\+\_\+\+Module()}{Udp\_Module()}}
{\footnotesize\ttfamily simple\+P2\+P\+::\+Udp\+\_\+\+Module\+::\+Udp\+\_\+\+Module (\begin{DoxyParamCaption}\item[{boost\+::asio\+::ip\+::address}]{broadcast\+\_\+address,  }\item[{Uint16}]{port,  }\item[{Uint32}]{beacon\+\_\+interval }\end{DoxyParamCaption})}



Constructor. 


\begin{DoxyParams}{Parameters}
{\em broadcast\+\_\+address} & address on which packets will be sent \\
\hline
{\em port} & port on which packets will be sent \\
\hline
{\em beacon\+\_\+interval} & beacon interval \\
\hline
\end{DoxyParams}


\subsubsection{Member Function Documentation}
\mbox{\Hypertarget{classsimpleP2P_1_1Udp__Module_a6e1598a38a557ebdf6eb4849b3cde9db}\label{classsimpleP2P_1_1Udp__Module_a6e1598a38a557ebdf6eb4849b3cde9db}} 
\index{simple\+P2\+P\+::\+Udp\+\_\+\+Module@{simple\+P2\+P\+::\+Udp\+\_\+\+Module}!init@{init}}
\index{init@{init}!simple\+P2\+P\+::\+Udp\+\_\+\+Module@{simple\+P2\+P\+::\+Udp\+\_\+\+Module}}
\paragraph{\texorpdfstring{init()}{init()}}
{\footnotesize\ttfamily std\+::thread simple\+P2\+P\+::\+Udp\+\_\+\+Module\+::init (\begin{DoxyParamCaption}{ }\end{DoxyParamCaption})}



Init methods run worker in thread and returns it. 

\begin{DoxyReturn}{Returns}
logging thread 
\end{DoxyReturn}
\mbox{\Hypertarget{classsimpleP2P_1_1Udp__Module_ad3904b2be4e113a8f7be11b23cfde4d2}\label{classsimpleP2P_1_1Udp__Module_ad3904b2be4e113a8f7be11b23cfde4d2}} 
\index{simple\+P2\+P\+::\+Udp\+\_\+\+Module@{simple\+P2\+P\+::\+Udp\+\_\+\+Module}!revoke\+\_\+file@{revoke\+\_\+file}}
\index{revoke\+\_\+file@{revoke\+\_\+file}!simple\+P2\+P\+::\+Udp\+\_\+\+Module@{simple\+P2\+P\+::\+Udp\+\_\+\+Module}}
\paragraph{\texorpdfstring{revoke\+\_\+file()}{revoke\_file()}}
{\footnotesize\ttfamily void simple\+P2\+P\+::\+Udp\+\_\+\+Module\+::revoke\+\_\+file (\begin{DoxyParamCaption}\item[{const \hyperlink{classsimpleP2P_1_1Resource}{Resource} \&}]{resource }\end{DoxyParamCaption})}



Sends revoke datagram. 


\begin{DoxyParams}{Parameters}
{\em resource} & \hyperlink{classsimpleP2P_1_1Resource}{Resource} t obe revoked \\
\hline
\end{DoxyParams}


The documentation for this class was generated from the following files\+:\begin{DoxyCompactItemize}
\item 
include/udp\+\_\+module.\+h\item 
src/udp\+\_\+module.\+cpp\end{DoxyCompactItemize}

\hypertarget{classsimpleP2P_1_1Resource}{}\section{simple\+P2P\+:\+:Resource Class Reference}
\label{classsimpleP2P_1_1Resource}\index{simple\+P2\+P\+::\+Resource@{simple\+P2\+P\+::\+Resource}}


Forward declaration.  




{\ttfamily \#include $<$resource.\+h$>$}

\subsection*{Public Member Functions}
\begin{DoxyCompactItemize}
\item 
\hyperlink{classsimpleP2P_1_1Resource_a0aed54e6cf7d3a5fa2c53fe3b3b57d19}{Resource} (std\+::string name, Uint64 size, std\+::string path=\char`\"{}./\char`\"{})
\begin{DoxyCompactList}\small\item\em Constructor. \end{DoxyCompactList}\item 
\hyperlink{classsimpleP2P_1_1Resource_aa725d8f5028c2f83a088a18bcbd9318d}{Resource} (std\+::vector$<$ Uint8 $>$ resource\+\_\+header)
\begin{DoxyCompactList}\small\item\em Constructor makes resource from header. \end{DoxyCompactList}\item 
std\+::vector$<$ Uint8 $>$ \hyperlink{classsimpleP2P_1_1Resource_a87e735b9e7b48c329698c6e7aab455a8}{generate\+\_\+resource\+\_\+header} ()
\begin{DoxyCompactList}\small\item\em Generates \hyperlink{classsimpleP2P_1_1Resource}{Resource} header. \end{DoxyCompactList}\item 
bool \hyperlink{classsimpleP2P_1_1Resource_ac3c067c66ef0db0a25a7904a83f50e3d}{has\+\_\+host} (\hyperlink{classsimpleP2P_1_1Host}{Host} host)
\begin{DoxyCompactList}\small\item\em Determines if resource is possesed by \hyperlink{classsimpleP2P_1_1Host}{Host}. \end{DoxyCompactList}\item 
Uint16 \hyperlink{classsimpleP2P_1_1Resource_a4bbfde4d1e25c62ae4da0e6dfd64900b}{calc\+\_\+segments\+\_\+count} () const
\begin{DoxyCompactList}\small\item\em Calculates and returns segment count. \end{DoxyCompactList}\item 
void \hyperlink{classsimpleP2P_1_1Resource_a49e46561e4088f78bb96c07c12d4738c}{set\+\_\+revoked} ()
\begin{DoxyCompactList}\small\item\em Function used to set invalidated flag. \end{DoxyCompactList}\item 
\mbox{\Hypertarget{classsimpleP2P_1_1Resource_a3b271172c4b67a378c264babbc1d72a1}\label{classsimpleP2P_1_1Resource_a3b271172c4b67a378c264babbc1d72a1}} 
bool {\bfseries is\+Invalidated} ()
\item 
Uint64 \hyperlink{classsimpleP2P_1_1Resource_adadeb491cccabbce2e1d883f9e8519f7}{get\+Size} () const
\begin{DoxyCompactList}\small\item\em Getter for file size. \end{DoxyCompactList}\item 
const std\+::string \& \hyperlink{classsimpleP2P_1_1Resource_adc12496aedf1729852d2c98bf94428aa}{get\+Name} () const
\begin{DoxyCompactList}\small\item\em Getter for file name. \end{DoxyCompactList}\item 
const std\+::string \& \hyperlink{classsimpleP2P_1_1Resource_a866cdd2e717abf3515629ca73b2f80b8}{get\+Path} () const
\begin{DoxyCompactList}\small\item\em Getter for file path. \end{DoxyCompactList}\item 
bool \hyperlink{classsimpleP2P_1_1Resource_a0b42735bed5ab425b9e26b660ededecf}{operator==} (const \hyperlink{classsimpleP2P_1_1Resource}{Resource} \&other) const
\begin{DoxyCompactList}\small\item\em Operator == checks file size and name for equality. \end{DoxyCompactList}\item 
bool \hyperlink{classsimpleP2P_1_1Resource_a5694c4c5a3d5b303a1fa0dcb3fb478b1}{operator!=} (const \hyperlink{classsimpleP2P_1_1Resource}{Resource} \&other) const
\begin{DoxyCompactList}\small\item\em Operator != checks file size and name for equality. \end{DoxyCompactList}\item 
\mbox{\Hypertarget{classsimpleP2P_1_1Resource_a4a8773417a0532826c75fdebe84304d3}\label{classsimpleP2P_1_1Resource_a4a8773417a0532826c75fdebe84304d3}} 
const tbb\+::concurrent\+\_\+vector$<$ std\+::weak\+\_\+ptr$<$ \hyperlink{classsimpleP2P_1_1Host}{Host} $>$ $>$ \& {\bfseries get\+\_\+hosts} () const
\end{DoxyCompactItemize}
\subsection*{Friends}
\begin{DoxyCompactItemize}
\item 
\mbox{\Hypertarget{classsimpleP2P_1_1Resource_a8d5b8c31f7b51293954816c91b76cabd}\label{classsimpleP2P_1_1Resource_a8d5b8c31f7b51293954816c91b76cabd}} 
class \hyperlink{classsimpleP2P_1_1Resource_a8d5b8c31f7b51293954816c91b76cabd}{Resource\+\_\+\+Database}
\begin{DoxyCompactList}\small\item\em friendship to manage \hyperlink{classsimpleP2P_1_1Resource}{Resource} Hosts, path etc \end{DoxyCompactList}\end{DoxyCompactItemize}


\subsection{Detailed Description}
Forward declaration. 

Class contains file information and points to nodes with file possesion 

\subsection{Constructor \& Destructor Documentation}
\mbox{\Hypertarget{classsimpleP2P_1_1Resource_a0aed54e6cf7d3a5fa2c53fe3b3b57d19}\label{classsimpleP2P_1_1Resource_a0aed54e6cf7d3a5fa2c53fe3b3b57d19}} 
\index{simple\+P2\+P\+::\+Resource@{simple\+P2\+P\+::\+Resource}!Resource@{Resource}}
\index{Resource@{Resource}!simple\+P2\+P\+::\+Resource@{simple\+P2\+P\+::\+Resource}}
\subsubsection{\texorpdfstring{Resource()}{Resource()}\hspace{0.1cm}{\footnotesize\ttfamily [1/2]}}
{\footnotesize\ttfamily simple\+P2\+P\+::\+Resource\+::\+Resource (\begin{DoxyParamCaption}\item[{std\+::string}]{name,  }\item[{Uint64}]{size,  }\item[{std\+::string}]{path = {\ttfamily \char`\"{}./\char`\"{}} }\end{DoxyParamCaption})}



Constructor. 


\begin{DoxyParams}{Parameters}
{\em name} & filename \\
\hline
{\em size} & filesize \\
\hline
{\em path} & filepath, default is \char`\"{}./\char`\"{} \\
\hline
\end{DoxyParams}
\mbox{\Hypertarget{classsimpleP2P_1_1Resource_aa725d8f5028c2f83a088a18bcbd9318d}\label{classsimpleP2P_1_1Resource_aa725d8f5028c2f83a088a18bcbd9318d}} 
\index{simple\+P2\+P\+::\+Resource@{simple\+P2\+P\+::\+Resource}!Resource@{Resource}}
\index{Resource@{Resource}!simple\+P2\+P\+::\+Resource@{simple\+P2\+P\+::\+Resource}}
\subsubsection{\texorpdfstring{Resource()}{Resource()}\hspace{0.1cm}{\footnotesize\ttfamily [2/2]}}
{\footnotesize\ttfamily simple\+P2\+P\+::\+Resource\+::\+Resource (\begin{DoxyParamCaption}\item[{std\+::vector$<$ Uint8 $>$}]{resource\+\_\+header }\end{DoxyParamCaption})}



Constructor makes resource from header. 


\begin{DoxyParams}{Parameters}
{\em resource\+\_\+header} & \hyperlink{classsimpleP2P_1_1Resource}{Resource} header \\
\hline
\end{DoxyParams}


\subsection{Member Function Documentation}
\mbox{\Hypertarget{classsimpleP2P_1_1Resource_a4bbfde4d1e25c62ae4da0e6dfd64900b}\label{classsimpleP2P_1_1Resource_a4bbfde4d1e25c62ae4da0e6dfd64900b}} 
\index{simple\+P2\+P\+::\+Resource@{simple\+P2\+P\+::\+Resource}!calc\+\_\+segments\+\_\+count@{calc\+\_\+segments\+\_\+count}}
\index{calc\+\_\+segments\+\_\+count@{calc\+\_\+segments\+\_\+count}!simple\+P2\+P\+::\+Resource@{simple\+P2\+P\+::\+Resource}}
\subsubsection{\texorpdfstring{calc\+\_\+segments\+\_\+count()}{calc\_segments\_count()}}
{\footnotesize\ttfamily Uint16 simple\+P2\+P\+::\+Resource\+::calc\+\_\+segments\+\_\+count (\begin{DoxyParamCaption}{ }\end{DoxyParamCaption}) const\hspace{0.3cm}{\ttfamily [inline]}}



Calculates and returns segment count. 

\begin{DoxyReturn}{Returns}
segment count 
\end{DoxyReturn}
\mbox{\Hypertarget{classsimpleP2P_1_1Resource_a87e735b9e7b48c329698c6e7aab455a8}\label{classsimpleP2P_1_1Resource_a87e735b9e7b48c329698c6e7aab455a8}} 
\index{simple\+P2\+P\+::\+Resource@{simple\+P2\+P\+::\+Resource}!generate\+\_\+resource\+\_\+header@{generate\+\_\+resource\+\_\+header}}
\index{generate\+\_\+resource\+\_\+header@{generate\+\_\+resource\+\_\+header}!simple\+P2\+P\+::\+Resource@{simple\+P2\+P\+::\+Resource}}
\subsubsection{\texorpdfstring{generate\+\_\+resource\+\_\+header()}{generate\_resource\_header()}}
{\footnotesize\ttfamily std\+::vector$<$ Uint8 $>$ simple\+P2\+P\+::\+Resource\+::generate\+\_\+resource\+\_\+header (\begin{DoxyParamCaption}{ }\end{DoxyParamCaption})}



Generates \hyperlink{classsimpleP2P_1_1Resource}{Resource} header. 

\begin{DoxyReturn}{Returns}
\hyperlink{classsimpleP2P_1_1Resource}{Resource} header 
\end{DoxyReturn}
\mbox{\Hypertarget{classsimpleP2P_1_1Resource_adc12496aedf1729852d2c98bf94428aa}\label{classsimpleP2P_1_1Resource_adc12496aedf1729852d2c98bf94428aa}} 
\index{simple\+P2\+P\+::\+Resource@{simple\+P2\+P\+::\+Resource}!get\+Name@{get\+Name}}
\index{get\+Name@{get\+Name}!simple\+P2\+P\+::\+Resource@{simple\+P2\+P\+::\+Resource}}
\subsubsection{\texorpdfstring{get\+Name()}{getName()}}
{\footnotesize\ttfamily const std\+::string \& simple\+P2\+P\+::\+Resource\+::get\+Name (\begin{DoxyParamCaption}{ }\end{DoxyParamCaption}) const}



Getter for file name. 

\begin{DoxyReturn}{Returns}
file name 
\end{DoxyReturn}
\mbox{\Hypertarget{classsimpleP2P_1_1Resource_a866cdd2e717abf3515629ca73b2f80b8}\label{classsimpleP2P_1_1Resource_a866cdd2e717abf3515629ca73b2f80b8}} 
\index{simple\+P2\+P\+::\+Resource@{simple\+P2\+P\+::\+Resource}!get\+Path@{get\+Path}}
\index{get\+Path@{get\+Path}!simple\+P2\+P\+::\+Resource@{simple\+P2\+P\+::\+Resource}}
\subsubsection{\texorpdfstring{get\+Path()}{getPath()}}
{\footnotesize\ttfamily const std\+::string \& simple\+P2\+P\+::\+Resource\+::get\+Path (\begin{DoxyParamCaption}{ }\end{DoxyParamCaption}) const}



Getter for file path. 

\begin{DoxyReturn}{Returns}
file path 
\end{DoxyReturn}
\mbox{\Hypertarget{classsimpleP2P_1_1Resource_adadeb491cccabbce2e1d883f9e8519f7}\label{classsimpleP2P_1_1Resource_adadeb491cccabbce2e1d883f9e8519f7}} 
\index{simple\+P2\+P\+::\+Resource@{simple\+P2\+P\+::\+Resource}!get\+Size@{get\+Size}}
\index{get\+Size@{get\+Size}!simple\+P2\+P\+::\+Resource@{simple\+P2\+P\+::\+Resource}}
\subsubsection{\texorpdfstring{get\+Size()}{getSize()}}
{\footnotesize\ttfamily Uint64 simple\+P2\+P\+::\+Resource\+::get\+Size (\begin{DoxyParamCaption}{ }\end{DoxyParamCaption}) const}



Getter for file size. 

\begin{DoxyReturn}{Returns}
file size 
\end{DoxyReturn}
\mbox{\Hypertarget{classsimpleP2P_1_1Resource_ac3c067c66ef0db0a25a7904a83f50e3d}\label{classsimpleP2P_1_1Resource_ac3c067c66ef0db0a25a7904a83f50e3d}} 
\index{simple\+P2\+P\+::\+Resource@{simple\+P2\+P\+::\+Resource}!has\+\_\+host@{has\+\_\+host}}
\index{has\+\_\+host@{has\+\_\+host}!simple\+P2\+P\+::\+Resource@{simple\+P2\+P\+::\+Resource}}
\subsubsection{\texorpdfstring{has\+\_\+host()}{has\_host()}}
{\footnotesize\ttfamily bool simple\+P2\+P\+::\+Resource\+::has\+\_\+host (\begin{DoxyParamCaption}\item[{\hyperlink{classsimpleP2P_1_1Host}{simple\+P2\+P\+::\+Host}}]{host }\end{DoxyParamCaption})}



Determines if resource is possesed by \hyperlink{classsimpleP2P_1_1Host}{Host}. 


\begin{DoxyParams}{Parameters}
{\em host} & \hyperlink{classsimpleP2P_1_1Host}{Host} \\
\hline
\end{DoxyParams}
\begin{DoxyReturn}{Returns}
true if resource is possessed by host 
\end{DoxyReturn}
\mbox{\Hypertarget{classsimpleP2P_1_1Resource_a5694c4c5a3d5b303a1fa0dcb3fb478b1}\label{classsimpleP2P_1_1Resource_a5694c4c5a3d5b303a1fa0dcb3fb478b1}} 
\index{simple\+P2\+P\+::\+Resource@{simple\+P2\+P\+::\+Resource}!operator"!=@{operator"!=}}
\index{operator"!=@{operator"!=}!simple\+P2\+P\+::\+Resource@{simple\+P2\+P\+::\+Resource}}
\subsubsection{\texorpdfstring{operator"!=()}{operator!=()}}
{\footnotesize\ttfamily bool simple\+P2\+P\+::\+Resource\+::operator!= (\begin{DoxyParamCaption}\item[{const \hyperlink{classsimpleP2P_1_1Resource}{Resource} \&}]{other }\end{DoxyParamCaption}) const}



Operator != checks file size and name for equality. 


\begin{DoxyParams}{Parameters}
{\em other} & other \\
\hline
\end{DoxyParams}
\begin{DoxyReturn}{Returns}
true if not equal 
\end{DoxyReturn}
\mbox{\Hypertarget{classsimpleP2P_1_1Resource_a0b42735bed5ab425b9e26b660ededecf}\label{classsimpleP2P_1_1Resource_a0b42735bed5ab425b9e26b660ededecf}} 
\index{simple\+P2\+P\+::\+Resource@{simple\+P2\+P\+::\+Resource}!operator==@{operator==}}
\index{operator==@{operator==}!simple\+P2\+P\+::\+Resource@{simple\+P2\+P\+::\+Resource}}
\subsubsection{\texorpdfstring{operator==()}{operator==()}}
{\footnotesize\ttfamily bool simple\+P2\+P\+::\+Resource\+::operator== (\begin{DoxyParamCaption}\item[{const \hyperlink{classsimpleP2P_1_1Resource}{Resource} \&}]{other }\end{DoxyParamCaption}) const}



Operator == checks file size and name for equality. 


\begin{DoxyParams}{Parameters}
{\em other} & other \\
\hline
\end{DoxyParams}
\begin{DoxyReturn}{Returns}
true if equal 
\end{DoxyReturn}
\mbox{\Hypertarget{classsimpleP2P_1_1Resource_a49e46561e4088f78bb96c07c12d4738c}\label{classsimpleP2P_1_1Resource_a49e46561e4088f78bb96c07c12d4738c}} 
\index{simple\+P2\+P\+::\+Resource@{simple\+P2\+P\+::\+Resource}!set\+\_\+revoked@{set\+\_\+revoked}}
\index{set\+\_\+revoked@{set\+\_\+revoked}!simple\+P2\+P\+::\+Resource@{simple\+P2\+P\+::\+Resource}}
\subsubsection{\texorpdfstring{set\+\_\+revoked()}{set\_revoked()}}
{\footnotesize\ttfamily void simple\+P2\+P\+::\+Resource\+::set\+\_\+revoked (\begin{DoxyParamCaption}{ }\end{DoxyParamCaption})\hspace{0.3cm}{\ttfamily [inline]}}



Function used to set invalidated flag. 

To allow references on resource outside database to gather information about revoke 

The documentation for this class was generated from the following files\+:\begin{DoxyCompactItemize}
\item 
include/resource.\+h\item 
src/resource.\+cpp\end{DoxyCompactItemize}

\hypertarget{classsimpleP2P_1_1Host}{}\subsection{simple\+P2P\+:\+:Host Class Reference}
\label{classsimpleP2P_1_1Host}\index{simple\+P2\+P\+::\+Host@{simple\+P2\+P\+::\+Host}}


Forward declaration.  




{\ttfamily \#include $<$host.\+h$>$}

\subsubsection*{Public Member Functions}
\begin{DoxyCompactItemize}
\item 
\hyperlink{classsimpleP2P_1_1Host_abbe0b5c51195b8cf2019d791ace5a5c7}{Host} (boost\+::asio\+::ip\+::address ip)
\begin{DoxyCompactList}\small\item\em Constructor. \end{DoxyCompactList}\item 
bool \hyperlink{classsimpleP2P_1_1Host_a5d4b48eaf05f5353816aae78cbd29c64}{has\+\_\+resource} (\hyperlink{classsimpleP2P_1_1Resource}{Resource} res)
\begin{DoxyCompactList}\small\item\em Determines if host has resource. \end{DoxyCompactList}\item 
bool \hyperlink{classsimpleP2P_1_1Host_aadac09c6ab516f62e5eebe28dd584626}{operator==} (const \hyperlink{classsimpleP2P_1_1Host}{Host} \&other) const
\begin{DoxyCompactList}\small\item\em Operator == checks host\+\_\+ip for equality. \end{DoxyCompactList}\item 
bool \hyperlink{classsimpleP2P_1_1Host_a13516e95bf59bb8dd6eea8940f8bb677}{operator!=} (const \hyperlink{classsimpleP2P_1_1Host}{Host} \&other) const
\begin{DoxyCompactList}\small\item\em Operator != checks host\+\_\+ip for equality. \end{DoxyCompactList}\end{DoxyCompactItemize}
\subsubsection*{Friends}
\begin{DoxyCompactItemize}
\item 
\mbox{\Hypertarget{classsimpleP2P_1_1Host_a8d5b8c31f7b51293954816c91b76cabd}\label{classsimpleP2P_1_1Host_a8d5b8c31f7b51293954816c91b76cabd}} 
class \hyperlink{classsimpleP2P_1_1Host_a8d5b8c31f7b51293954816c91b76cabd}{Resource\+\_\+\+Database}
\begin{DoxyCompactList}\small\item\em friendship to manage \hyperlink{classsimpleP2P_1_1Host}{Host}\textquotesingle{}s Resources timeouts etc \end{DoxyCompactList}\end{DoxyCompactItemize}


\subsubsection{Detailed Description}
Forward declaration. 

Class contains node information and points to files it possess 

\subsubsection{Constructor \& Destructor Documentation}
\mbox{\Hypertarget{classsimpleP2P_1_1Host_abbe0b5c51195b8cf2019d791ace5a5c7}\label{classsimpleP2P_1_1Host_abbe0b5c51195b8cf2019d791ace5a5c7}} 
\index{simple\+P2\+P\+::\+Host@{simple\+P2\+P\+::\+Host}!Host@{Host}}
\index{Host@{Host}!simple\+P2\+P\+::\+Host@{simple\+P2\+P\+::\+Host}}
\paragraph{\texorpdfstring{Host()}{Host()}}
{\footnotesize\ttfamily simple\+P2\+P\+::\+Host\+::\+Host (\begin{DoxyParamCaption}\item[{boost\+::asio\+::ip\+::address}]{ip }\end{DoxyParamCaption})}



Constructor. 


\begin{DoxyParams}{Parameters}
{\em ip} & Ip of the \hyperlink{classsimpleP2P_1_1Host}{Host} \\
\hline
\end{DoxyParams}


\subsubsection{Member Function Documentation}
\mbox{\Hypertarget{classsimpleP2P_1_1Host_a5d4b48eaf05f5353816aae78cbd29c64}\label{classsimpleP2P_1_1Host_a5d4b48eaf05f5353816aae78cbd29c64}} 
\index{simple\+P2\+P\+::\+Host@{simple\+P2\+P\+::\+Host}!has\+\_\+resource@{has\+\_\+resource}}
\index{has\+\_\+resource@{has\+\_\+resource}!simple\+P2\+P\+::\+Host@{simple\+P2\+P\+::\+Host}}
\paragraph{\texorpdfstring{has\+\_\+resource()}{has\_resource()}}
{\footnotesize\ttfamily bool simple\+P2\+P\+::\+Host\+::has\+\_\+resource (\begin{DoxyParamCaption}\item[{\hyperlink{classsimpleP2P_1_1Resource}{Resource}}]{res }\end{DoxyParamCaption})}



Determines if host has resource. 


\begin{DoxyParams}{Parameters}
{\em res} & \hyperlink{classsimpleP2P_1_1Resource}{Resource} to be checked \\
\hline
\end{DoxyParams}
\begin{DoxyReturn}{Returns}
true if \hyperlink{classsimpleP2P_1_1Host}{Host} has \hyperlink{classsimpleP2P_1_1Resource}{Resource} res 
\end{DoxyReturn}
\mbox{\Hypertarget{classsimpleP2P_1_1Host_a13516e95bf59bb8dd6eea8940f8bb677}\label{classsimpleP2P_1_1Host_a13516e95bf59bb8dd6eea8940f8bb677}} 
\index{simple\+P2\+P\+::\+Host@{simple\+P2\+P\+::\+Host}!operator"!=@{operator"!=}}
\index{operator"!=@{operator"!=}!simple\+P2\+P\+::\+Host@{simple\+P2\+P\+::\+Host}}
\paragraph{\texorpdfstring{operator"!=()}{operator!=()}}
{\footnotesize\ttfamily bool simple\+P2\+P\+::\+Host\+::operator!= (\begin{DoxyParamCaption}\item[{const \hyperlink{classsimpleP2P_1_1Host}{Host} \&}]{other }\end{DoxyParamCaption}) const}



Operator != checks host\+\_\+ip for equality. 


\begin{DoxyParams}{Parameters}
{\em other} & other \\
\hline
\end{DoxyParams}
\begin{DoxyReturn}{Returns}
true if not equal 
\end{DoxyReturn}
\mbox{\Hypertarget{classsimpleP2P_1_1Host_aadac09c6ab516f62e5eebe28dd584626}\label{classsimpleP2P_1_1Host_aadac09c6ab516f62e5eebe28dd584626}} 
\index{simple\+P2\+P\+::\+Host@{simple\+P2\+P\+::\+Host}!operator==@{operator==}}
\index{operator==@{operator==}!simple\+P2\+P\+::\+Host@{simple\+P2\+P\+::\+Host}}
\paragraph{\texorpdfstring{operator==()}{operator==()}}
{\footnotesize\ttfamily bool simple\+P2\+P\+::\+Host\+::operator== (\begin{DoxyParamCaption}\item[{const \hyperlink{classsimpleP2P_1_1Host}{Host} \&}]{other }\end{DoxyParamCaption}) const}



Operator == checks host\+\_\+ip for equality. 


\begin{DoxyParams}{Parameters}
{\em other} & other \\
\hline
\end{DoxyParams}
\begin{DoxyReturn}{Returns}
true if equal 
\end{DoxyReturn}


The documentation for this class was generated from the following files\+:\begin{DoxyCompactItemize}
\item 
include/host.\+h\item 
src/host.\+cpp\end{DoxyCompactItemize}

\hypertarget{classsimpleP2P_1_1Resource__Database}{}\section{simple\+P2P\+:\+:Resource\+\_\+\+Database Class Reference}
\label{classsimpleP2P_1_1Resource__Database}\index{simple\+P2\+P\+::\+Resource\+\_\+\+Database@{simple\+P2\+P\+::\+Resource\+\_\+\+Database}}


Class holding information about files in network and on localhost.  




{\ttfamily \#include $<$resource\+\_\+database.\+h$>$}

\subsection*{Public Member Functions}
\begin{DoxyCompactItemize}
\item 
\hyperlink{classsimpleP2P_1_1Resource__Database_a0158b1b1c4583d60534c2bea5dc437a3}{Resource\+\_\+\+Database} (\hyperlink{classsimpleP2P_1_1Host}{Host} localhost)
\begin{DoxyCompactList}\small\item\em Constructor. \end{DoxyCompactList}\item 
bool \hyperlink{classsimpleP2P_1_1Resource__Database_a2a6cd0b6da11176aec9ef7f218ac1c62}{has\+\_\+file} (const \hyperlink{classsimpleP2P_1_1Resource}{Resource} \&res)
\begin{DoxyCompactList}\small\item\em Check if localhost has certain file. \end{DoxyCompactList}\item 
void \hyperlink{classsimpleP2P_1_1Resource__Database_a7afdd297b15fb225b61f9d52e2ab9cc6}{add\+\_\+file} (const \hyperlink{classsimpleP2P_1_1Resource}{Resource} \&res, const \hyperlink{classsimpleP2P_1_1Host}{Host} \&host)
\begin{DoxyCompactList}\small\item\em Adds connection between file and resource, adn creates them if they do not exist. \end{DoxyCompactList}\item 
bool \hyperlink{classsimpleP2P_1_1Resource__Database_a9b0fe1012eba333db6d8e99b02764d4d}{remove\+\_\+file} (const \hyperlink{classsimpleP2P_1_1Resource}{Resource} \&res, const \hyperlink{classsimpleP2P_1_1Host}{Host} \&host)
\begin{DoxyCompactList}\small\item\em Removes connection between file and resource. \end{DoxyCompactList}\item 
void \hyperlink{classsimpleP2P_1_1Resource__Database_a181e862da4be11acbdb7d5569ad6925a}{update\+\_\+host} (const \hyperlink{classsimpleP2P_1_1Host}{Host} \&host)
\begin{DoxyCompactList}\small\item\em Updates the list of resources aviable from host Triggered after receive of full Beacon Packet. \end{DoxyCompactList}\item 
void \hyperlink{classsimpleP2P_1_1Resource__Database_a7ea91d496814c3d911b14ef4cd863943}{revoke\+\_\+resource} (const \hyperlink{classsimpleP2P_1_1Resource}{Resource} \&resource)
\begin{DoxyCompactList}\small\item\em Revokes resource and disconnects it from Hosts in database and database itself \hyperlink{classsimpleP2P_1_1Resource}{Resource} will still point to Hosts that possess it. \end{DoxyCompactList}\item 
void \hyperlink{classsimpleP2P_1_1Resource__Database_a8ac7f42f7de6f3281974d2be7f1ea8e8}{add\+\_\+file} (const \hyperlink{classsimpleP2P_1_1Resource}{Resource} \&res)
\begin{DoxyCompactList}\small\item\em same as add\+\_\+file(\+Resource, Host) but host is localhost \end{DoxyCompactList}\item 
bool \hyperlink{classsimpleP2P_1_1Resource__Database_aa7e7b4575b8217dc3e52fb741ff0474a}{remove\+\_\+file} (const \hyperlink{classsimpleP2P_1_1Resource}{Resource} \&res)
\begin{DoxyCompactList}\small\item\em same as remove\+\_\+file(\+Resource, Host) but host is localhost \end{DoxyCompactList}\item 
std\+::shared\+\_\+ptr$<$ \hyperlink{classsimpleP2P_1_1Resource}{Resource} $>$ \hyperlink{classsimpleP2P_1_1Resource__Database_ac1914413c36b5b1cf343aac6e9ba7a6f}{who\+\_\+has\+\_\+file} (std\+::vector$<$ Uint8 $>$ resource\+\_\+header)
\begin{DoxyCompactList}\small\item\em Returns shared pointer to resource to allow access to information about file owners. \end{DoxyCompactList}\item 
std\+::shared\+\_\+ptr$<$ \hyperlink{classsimpleP2P_1_1Resource}{Resource} $>$ \hyperlink{classsimpleP2P_1_1Resource__Database_a70aa9503db5700a6abdfdc9b029d71c1}{who\+\_\+has\+\_\+file} (const \hyperlink{classsimpleP2P_1_1Resource}{Resource} \&res)
\begin{DoxyCompactList}\small\item\em Returns shared pointer to resource to allow access to information about file owners. \end{DoxyCompactList}\item 
std\+::vector$<$ std\+::vector$<$ Uint8 $>$ $>$ \hyperlink{classsimpleP2P_1_1Resource__Database_a77b35fd391868164073005ffbd0377ae}{generate\+\_\+database\+\_\+headers} ()
\begin{DoxyCompactList}\small\item\em Generates listing of localhost content in a header. \end{DoxyCompactList}\item 
std\+::shared\+\_\+ptr$<$ \hyperlink{classsimpleP2P_1_1Host}{Host} $>$ \hyperlink{classsimpleP2P_1_1Resource__Database_a3d3e448f6b0502129e68d42214e1a059}{get\+Host} () const
\begin{DoxyCompactList}\small\item\em Get localhost information. \end{DoxyCompactList}\item 
\mbox{\Hypertarget{classsimpleP2P_1_1Resource__Database_a97b8e423121d50cf2593ccd4cfef4ca9}\label{classsimpleP2P_1_1Resource__Database_a97b8e423121d50cf2593ccd4cfef4ca9}} 
const std\+::vector$<$ std\+::shared\+\_\+ptr$<$ \hyperlink{classsimpleP2P_1_1Resource}{Resource} $>$ $>$ {\bfseries get\+Resources} () const
\end{DoxyCompactItemize}


\subsection{Detailed Description}
Class holding information about files in network and on localhost. 

\subsection{Constructor \& Destructor Documentation}
\mbox{\Hypertarget{classsimpleP2P_1_1Resource__Database_a0158b1b1c4583d60534c2bea5dc437a3}\label{classsimpleP2P_1_1Resource__Database_a0158b1b1c4583d60534c2bea5dc437a3}} 
\index{simple\+P2\+P\+::\+Resource\+\_\+\+Database@{simple\+P2\+P\+::\+Resource\+\_\+\+Database}!Resource\+\_\+\+Database@{Resource\+\_\+\+Database}}
\index{Resource\+\_\+\+Database@{Resource\+\_\+\+Database}!simple\+P2\+P\+::\+Resource\+\_\+\+Database@{simple\+P2\+P\+::\+Resource\+\_\+\+Database}}
\subsubsection{\texorpdfstring{Resource\+\_\+\+Database()}{Resource\_Database()}}
{\footnotesize\ttfamily simple\+P2\+P\+::\+Resource\+\_\+\+Database\+::\+Resource\+\_\+\+Database (\begin{DoxyParamCaption}\item[{\hyperlink{classsimpleP2P_1_1Host}{Host}}]{localhost }\end{DoxyParamCaption})}



Constructor. 


\begin{DoxyParams}{Parameters}
{\em localhost} & localhost \\
\hline
\end{DoxyParams}


\subsection{Member Function Documentation}
\mbox{\Hypertarget{classsimpleP2P_1_1Resource__Database_a7afdd297b15fb225b61f9d52e2ab9cc6}\label{classsimpleP2P_1_1Resource__Database_a7afdd297b15fb225b61f9d52e2ab9cc6}} 
\index{simple\+P2\+P\+::\+Resource\+\_\+\+Database@{simple\+P2\+P\+::\+Resource\+\_\+\+Database}!add\+\_\+file@{add\+\_\+file}}
\index{add\+\_\+file@{add\+\_\+file}!simple\+P2\+P\+::\+Resource\+\_\+\+Database@{simple\+P2\+P\+::\+Resource\+\_\+\+Database}}
\subsubsection{\texorpdfstring{add\+\_\+file()}{add\_file()}\hspace{0.1cm}{\footnotesize\ttfamily [1/2]}}
{\footnotesize\ttfamily void simple\+P2\+P\+::\+Resource\+\_\+\+Database\+::add\+\_\+file (\begin{DoxyParamCaption}\item[{const \hyperlink{classsimpleP2P_1_1Resource}{Resource} \&}]{res,  }\item[{const \hyperlink{classsimpleP2P_1_1Host}{Host} \&}]{host }\end{DoxyParamCaption})}



Adds connection between file and resource, adn creates them if they do not exist. 


\begin{DoxyParams}{Parameters}
{\em res} & \hyperlink{classsimpleP2P_1_1Resource}{Resource} to be added \\
\hline
{\em host} & \hyperlink{classsimpleP2P_1_1Host}{Host} which possess \hyperlink{classsimpleP2P_1_1Resource}{Resource} res \\
\hline
\end{DoxyParams}
\mbox{\Hypertarget{classsimpleP2P_1_1Resource__Database_a8ac7f42f7de6f3281974d2be7f1ea8e8}\label{classsimpleP2P_1_1Resource__Database_a8ac7f42f7de6f3281974d2be7f1ea8e8}} 
\index{simple\+P2\+P\+::\+Resource\+\_\+\+Database@{simple\+P2\+P\+::\+Resource\+\_\+\+Database}!add\+\_\+file@{add\+\_\+file}}
\index{add\+\_\+file@{add\+\_\+file}!simple\+P2\+P\+::\+Resource\+\_\+\+Database@{simple\+P2\+P\+::\+Resource\+\_\+\+Database}}
\subsubsection{\texorpdfstring{add\+\_\+file()}{add\_file()}\hspace{0.1cm}{\footnotesize\ttfamily [2/2]}}
{\footnotesize\ttfamily void simple\+P2\+P\+::\+Resource\+\_\+\+Database\+::add\+\_\+file (\begin{DoxyParamCaption}\item[{const \hyperlink{classsimpleP2P_1_1Resource}{Resource} \&}]{res }\end{DoxyParamCaption})}



same as add\+\_\+file(\+Resource, Host) but host is localhost 


\begin{DoxyParams}{Parameters}
{\em res} & \hyperlink{classsimpleP2P_1_1Resource}{Resource} to be added \\
\hline
\end{DoxyParams}
\mbox{\Hypertarget{classsimpleP2P_1_1Resource__Database_a77b35fd391868164073005ffbd0377ae}\label{classsimpleP2P_1_1Resource__Database_a77b35fd391868164073005ffbd0377ae}} 
\index{simple\+P2\+P\+::\+Resource\+\_\+\+Database@{simple\+P2\+P\+::\+Resource\+\_\+\+Database}!generate\+\_\+database\+\_\+headers@{generate\+\_\+database\+\_\+headers}}
\index{generate\+\_\+database\+\_\+headers@{generate\+\_\+database\+\_\+headers}!simple\+P2\+P\+::\+Resource\+\_\+\+Database@{simple\+P2\+P\+::\+Resource\+\_\+\+Database}}
\subsubsection{\texorpdfstring{generate\+\_\+database\+\_\+headers()}{generate\_database\_headers()}}
{\footnotesize\ttfamily std\+::vector$<$ std\+::vector$<$ Uint8 $>$ $>$ simple\+P2\+P\+::\+Resource\+\_\+\+Database\+::generate\+\_\+database\+\_\+headers (\begin{DoxyParamCaption}{ }\end{DoxyParamCaption})}



Generates listing of localhost content in a header. 

\begin{DoxyReturn}{Returns}
listing header of localhost resources 
\end{DoxyReturn}
\mbox{\Hypertarget{classsimpleP2P_1_1Resource__Database_a3d3e448f6b0502129e68d42214e1a059}\label{classsimpleP2P_1_1Resource__Database_a3d3e448f6b0502129e68d42214e1a059}} 
\index{simple\+P2\+P\+::\+Resource\+\_\+\+Database@{simple\+P2\+P\+::\+Resource\+\_\+\+Database}!get\+Host@{get\+Host}}
\index{get\+Host@{get\+Host}!simple\+P2\+P\+::\+Resource\+\_\+\+Database@{simple\+P2\+P\+::\+Resource\+\_\+\+Database}}
\subsubsection{\texorpdfstring{get\+Host()}{getHost()}}
{\footnotesize\ttfamily std\+::shared\+\_\+ptr$<$ \hyperlink{classsimpleP2P_1_1Host}{Host} $>$ simple\+P2\+P\+::\+Resource\+\_\+\+Database\+::get\+Host (\begin{DoxyParamCaption}{ }\end{DoxyParamCaption}) const}



Get localhost information. 

\begin{DoxyReturn}{Returns}
localhost 
\end{DoxyReturn}
\mbox{\Hypertarget{classsimpleP2P_1_1Resource__Database_a2a6cd0b6da11176aec9ef7f218ac1c62}\label{classsimpleP2P_1_1Resource__Database_a2a6cd0b6da11176aec9ef7f218ac1c62}} 
\index{simple\+P2\+P\+::\+Resource\+\_\+\+Database@{simple\+P2\+P\+::\+Resource\+\_\+\+Database}!has\+\_\+file@{has\+\_\+file}}
\index{has\+\_\+file@{has\+\_\+file}!simple\+P2\+P\+::\+Resource\+\_\+\+Database@{simple\+P2\+P\+::\+Resource\+\_\+\+Database}}
\subsubsection{\texorpdfstring{has\+\_\+file()}{has\_file()}}
{\footnotesize\ttfamily bool simple\+P2\+P\+::\+Resource\+\_\+\+Database\+::has\+\_\+file (\begin{DoxyParamCaption}\item[{const \hyperlink{classsimpleP2P_1_1Resource}{Resource} \&}]{res }\end{DoxyParamCaption})}



Check if localhost has certain file. 


\begin{DoxyParams}{Parameters}
{\em res} & \hyperlink{classsimpleP2P_1_1Resource}{Resource} to be checked \\
\hline
\end{DoxyParams}
\begin{DoxyReturn}{Returns}
true if host already has some resource 
\end{DoxyReturn}
\mbox{\Hypertarget{classsimpleP2P_1_1Resource__Database_a9b0fe1012eba333db6d8e99b02764d4d}\label{classsimpleP2P_1_1Resource__Database_a9b0fe1012eba333db6d8e99b02764d4d}} 
\index{simple\+P2\+P\+::\+Resource\+\_\+\+Database@{simple\+P2\+P\+::\+Resource\+\_\+\+Database}!remove\+\_\+file@{remove\+\_\+file}}
\index{remove\+\_\+file@{remove\+\_\+file}!simple\+P2\+P\+::\+Resource\+\_\+\+Database@{simple\+P2\+P\+::\+Resource\+\_\+\+Database}}
\subsubsection{\texorpdfstring{remove\+\_\+file()}{remove\_file()}\hspace{0.1cm}{\footnotesize\ttfamily [1/2]}}
{\footnotesize\ttfamily bool simple\+P2\+P\+::\+Resource\+\_\+\+Database\+::remove\+\_\+file (\begin{DoxyParamCaption}\item[{const \hyperlink{classsimpleP2P_1_1Resource}{Resource} \&}]{res,  }\item[{const \hyperlink{classsimpleP2P_1_1Host}{Host} \&}]{host }\end{DoxyParamCaption})}



Removes connection between file and resource. 


\begin{DoxyParams}{Parameters}
{\em res} & \hyperlink{classsimpleP2P_1_1Resource}{Resource} to be removed from host list \\
\hline
{\em host} & \hyperlink{classsimpleP2P_1_1Host}{Host} which resource will be removed \\
\hline
\end{DoxyParams}
\begin{DoxyReturn}{Returns}
returns false if file did not existed or was not possesed 
\end{DoxyReturn}
\mbox{\Hypertarget{classsimpleP2P_1_1Resource__Database_aa7e7b4575b8217dc3e52fb741ff0474a}\label{classsimpleP2P_1_1Resource__Database_aa7e7b4575b8217dc3e52fb741ff0474a}} 
\index{simple\+P2\+P\+::\+Resource\+\_\+\+Database@{simple\+P2\+P\+::\+Resource\+\_\+\+Database}!remove\+\_\+file@{remove\+\_\+file}}
\index{remove\+\_\+file@{remove\+\_\+file}!simple\+P2\+P\+::\+Resource\+\_\+\+Database@{simple\+P2\+P\+::\+Resource\+\_\+\+Database}}
\subsubsection{\texorpdfstring{remove\+\_\+file()}{remove\_file()}\hspace{0.1cm}{\footnotesize\ttfamily [2/2]}}
{\footnotesize\ttfamily bool simple\+P2\+P\+::\+Resource\+\_\+\+Database\+::remove\+\_\+file (\begin{DoxyParamCaption}\item[{const \hyperlink{classsimpleP2P_1_1Resource}{Resource} \&}]{res }\end{DoxyParamCaption})}



same as remove\+\_\+file(\+Resource, Host) but host is localhost 


\begin{DoxyParams}{Parameters}
{\em res} & \hyperlink{classsimpleP2P_1_1Resource}{Resource} to be removed from localhost list \\
\hline
\end{DoxyParams}
\begin{DoxyReturn}{Returns}
returns false if file did not existed or was not possesed 
\end{DoxyReturn}
\mbox{\Hypertarget{classsimpleP2P_1_1Resource__Database_a7ea91d496814c3d911b14ef4cd863943}\label{classsimpleP2P_1_1Resource__Database_a7ea91d496814c3d911b14ef4cd863943}} 
\index{simple\+P2\+P\+::\+Resource\+\_\+\+Database@{simple\+P2\+P\+::\+Resource\+\_\+\+Database}!revoke\+\_\+resource@{revoke\+\_\+resource}}
\index{revoke\+\_\+resource@{revoke\+\_\+resource}!simple\+P2\+P\+::\+Resource\+\_\+\+Database@{simple\+P2\+P\+::\+Resource\+\_\+\+Database}}
\subsubsection{\texorpdfstring{revoke\+\_\+resource()}{revoke\_resource()}}
{\footnotesize\ttfamily void simple\+P2\+P\+::\+Resource\+\_\+\+Database\+::revoke\+\_\+resource (\begin{DoxyParamCaption}\item[{const \hyperlink{classsimpleP2P_1_1Resource}{Resource} \&}]{resource }\end{DoxyParamCaption})}



Revokes resource and disconnects it from Hosts in database and database itself \hyperlink{classsimpleP2P_1_1Resource}{Resource} will still point to Hosts that possess it. 


\begin{DoxyParams}{Parameters}
{\em resource} & \hyperlink{classsimpleP2P_1_1Resource}{Resource} to be revoked \\
\hline
\end{DoxyParams}
\mbox{\Hypertarget{classsimpleP2P_1_1Resource__Database_a181e862da4be11acbdb7d5569ad6925a}\label{classsimpleP2P_1_1Resource__Database_a181e862da4be11acbdb7d5569ad6925a}} 
\index{simple\+P2\+P\+::\+Resource\+\_\+\+Database@{simple\+P2\+P\+::\+Resource\+\_\+\+Database}!update\+\_\+host@{update\+\_\+host}}
\index{update\+\_\+host@{update\+\_\+host}!simple\+P2\+P\+::\+Resource\+\_\+\+Database@{simple\+P2\+P\+::\+Resource\+\_\+\+Database}}
\subsubsection{\texorpdfstring{update\+\_\+host()}{update\_host()}}
{\footnotesize\ttfamily void simple\+P2\+P\+::\+Resource\+\_\+\+Database\+::update\+\_\+host (\begin{DoxyParamCaption}\item[{const \hyperlink{classsimpleP2P_1_1Host}{Host} \&}]{host }\end{DoxyParamCaption})}



Updates the list of resources aviable from host Triggered after receive of full Beacon Packet. 


\begin{DoxyParams}{Parameters}
{\em host} & \hyperlink{classsimpleP2P_1_1Host}{Host} and possesed resources in a struct \\
\hline
\end{DoxyParams}
\mbox{\Hypertarget{classsimpleP2P_1_1Resource__Database_ac1914413c36b5b1cf343aac6e9ba7a6f}\label{classsimpleP2P_1_1Resource__Database_ac1914413c36b5b1cf343aac6e9ba7a6f}} 
\index{simple\+P2\+P\+::\+Resource\+\_\+\+Database@{simple\+P2\+P\+::\+Resource\+\_\+\+Database}!who\+\_\+has\+\_\+file@{who\+\_\+has\+\_\+file}}
\index{who\+\_\+has\+\_\+file@{who\+\_\+has\+\_\+file}!simple\+P2\+P\+::\+Resource\+\_\+\+Database@{simple\+P2\+P\+::\+Resource\+\_\+\+Database}}
\subsubsection{\texorpdfstring{who\+\_\+has\+\_\+file()}{who\_has\_file()}\hspace{0.1cm}{\footnotesize\ttfamily [1/2]}}
{\footnotesize\ttfamily shared\+\_\+ptr$<$ \hyperlink{classsimpleP2P_1_1Resource}{Resource} $>$ simple\+P2\+P\+::\+Resource\+\_\+\+Database\+::who\+\_\+has\+\_\+file (\begin{DoxyParamCaption}\item[{std\+::vector$<$ Uint8 $>$}]{resource\+\_\+header }\end{DoxyParamCaption})\hspace{0.3cm}{\ttfamily [inline]}}



Returns shared pointer to resource to allow access to information about file owners. 


\begin{DoxyParams}{Parameters}
{\em res} & \hyperlink{classsimpleP2P_1_1Resource}{Resource} about which information is gathered \\
\hline
\end{DoxyParams}
\begin{DoxyReturn}{Returns}
shared pointer to res 
\end{DoxyReturn}
\mbox{\Hypertarget{classsimpleP2P_1_1Resource__Database_a70aa9503db5700a6abdfdc9b029d71c1}\label{classsimpleP2P_1_1Resource__Database_a70aa9503db5700a6abdfdc9b029d71c1}} 
\index{simple\+P2\+P\+::\+Resource\+\_\+\+Database@{simple\+P2\+P\+::\+Resource\+\_\+\+Database}!who\+\_\+has\+\_\+file@{who\+\_\+has\+\_\+file}}
\index{who\+\_\+has\+\_\+file@{who\+\_\+has\+\_\+file}!simple\+P2\+P\+::\+Resource\+\_\+\+Database@{simple\+P2\+P\+::\+Resource\+\_\+\+Database}}
\subsubsection{\texorpdfstring{who\+\_\+has\+\_\+file()}{who\_has\_file()}\hspace{0.1cm}{\footnotesize\ttfamily [2/2]}}
{\footnotesize\ttfamily shared\+\_\+ptr$<$ \hyperlink{classsimpleP2P_1_1Resource}{Resource} $>$ simple\+P2\+P\+::\+Resource\+\_\+\+Database\+::who\+\_\+has\+\_\+file (\begin{DoxyParamCaption}\item[{const \hyperlink{classsimpleP2P_1_1Resource}{Resource} \&}]{res }\end{DoxyParamCaption})}



Returns shared pointer to resource to allow access to information about file owners. 


\begin{DoxyParams}{Parameters}
{\em res} & \hyperlink{classsimpleP2P_1_1Resource}{Resource} about which information is gathered \\
\hline
\end{DoxyParams}
\begin{DoxyReturn}{Returns}
shared pointer to res 
\end{DoxyReturn}


The documentation for this class was generated from the following files\+:\begin{DoxyCompactItemize}
\item 
include/resource\+\_\+database.\+h\item 
src/resource\+\_\+database.\+cpp\end{DoxyCompactItemize}

\hypertarget{classSimpleP2P_1_1FileManager}{}\section{Simple\+P2P\+:\+:File\+Manager Class Reference}
\label{classSimpleP2P_1_1FileManager}\index{Simple\+P2\+P\+::\+File\+Manager@{Simple\+P2\+P\+::\+File\+Manager}}


Handles read/write to the files on disc.  




{\ttfamily \#include $<$File\+Manager.\+h$>$}

\subsection*{Public Member Functions}
\begin{DoxyCompactItemize}
\item 
void \hyperlink{classSimpleP2P_1_1FileManager_a9f3fd7a6b4c695b79258448dc60b0b66}{get\+\_\+file} (\hyperlink{classSimpleP2P_1_1FileRequest}{File\+Request} request, char $\ast$result, std\+::size\+\_\+t size)
\begin{DoxyCompactList}\small\item\em Buffers specificated segments of the specificated file in the char array. \end{DoxyCompactList}\item 
void \hyperlink{classSimpleP2P_1_1FileManager_a6bc1cab4cfac8c75186147b9bf1f29e3}{store\+\_\+resource} (Complete\+Resource \&resource)
\begin{DoxyCompactList}\small\item\em Stores the file contents in the physical file on disc. \end{DoxyCompactList}\end{DoxyCompactItemize}


\subsection{Detailed Description}
Handles read/write to the files on disc. 

An A\+PI which provides\+:
\begin{DoxyItemize}
\item buffering contents of requested segments of a specificated local file,
\item storing a complete, downloaded file physically on the local disc. Ensures synchronization of those operations. 
\end{DoxyItemize}

\subsection{Member Function Documentation}
\mbox{\Hypertarget{classSimpleP2P_1_1FileManager_a9f3fd7a6b4c695b79258448dc60b0b66}\label{classSimpleP2P_1_1FileManager_a9f3fd7a6b4c695b79258448dc60b0b66}} 
\index{Simple\+P2\+P\+::\+File\+Manager@{Simple\+P2\+P\+::\+File\+Manager}!get\+\_\+file@{get\+\_\+file}}
\index{get\+\_\+file@{get\+\_\+file}!Simple\+P2\+P\+::\+File\+Manager@{Simple\+P2\+P\+::\+File\+Manager}}
\subsubsection{\texorpdfstring{get\+\_\+file()}{get\_file()}}
{\footnotesize\ttfamily void Simple\+P2\+P\+::\+File\+Manager\+::get\+\_\+file (\begin{DoxyParamCaption}\item[{\hyperlink{classSimpleP2P_1_1FileRequest}{File\+Request}}]{request,  }\item[{char $\ast$}]{result,  }\item[{std\+::size\+\_\+t}]{size }\end{DoxyParamCaption})}



Buffers specificated segments of the specificated file in the char array. 

(!) All segments will be returned concatenated in a single char array, provided in the \textquotesingle{}result\textquotesingle{} parameter. They will be put to the array in the order as provided in the \textquotesingle{}request\textquotesingle{} param. Keep this in mind if you requested the last segment of the file, size of which may vary.


\begin{DoxyParams}{Parameters}
{\em request} & Specifies file and its segments to buffer. \\
\hline
{\em result} & The array to buffer the file contents in. \\
\hline
{\em size} & Size of the char array. \\
\hline
\end{DoxyParams}
\mbox{\Hypertarget{classSimpleP2P_1_1FileManager_a6bc1cab4cfac8c75186147b9bf1f29e3}\label{classSimpleP2P_1_1FileManager_a6bc1cab4cfac8c75186147b9bf1f29e3}} 
\index{Simple\+P2\+P\+::\+File\+Manager@{Simple\+P2\+P\+::\+File\+Manager}!store\+\_\+resource@{store\+\_\+resource}}
\index{store\+\_\+resource@{store\+\_\+resource}!Simple\+P2\+P\+::\+File\+Manager@{Simple\+P2\+P\+::\+File\+Manager}}
\subsubsection{\texorpdfstring{store\+\_\+resource()}{store\_resource()}}
{\footnotesize\ttfamily void Simple\+P2\+P\+::\+File\+Manager\+::store\+\_\+resource (\begin{DoxyParamCaption}\item[{Complete\+Resource \&}]{resource }\end{DoxyParamCaption})}



Stores the file contents in the physical file on disc. 


\begin{DoxyParams}{Parameters}
{\em resource} & File to store on the disc. The data will not be interpreted, so make sure it\textquotesingle{}s complete and ready to store. \\
\hline
\end{DoxyParams}


The documentation for this class was generated from the following files\+:\begin{DoxyCompactItemize}
\item 
include/File\+Manager.\+h\item 
src/File\+Manager.\+cpp\end{DoxyCompactItemize}

\hypertarget{classSimpleP2P_1_1FileRequest}{}\subsection{Simple\+P2P\+:\+:File\+Request Class Reference}
\label{classSimpleP2P_1_1FileRequest}\index{Simple\+P2\+P\+::\+File\+Request@{Simple\+P2\+P\+::\+File\+Request}}


Carries info about a single file transfer request -\/ resource header and numbers of wanted segments.  




{\ttfamily \#include $<$File\+Request.\+h$>$}

\subsubsection*{Public Member Functions}
\begin{DoxyCompactItemize}
\item 
\hyperlink{classSimpleP2P_1_1FileRequest_a6e50d4f7d07a46ddcd72745ebc6e9e81}{File\+Request} (std\+::vector$<$ Int8 $>$ rh, std\+::initializer\+\_\+list$<$ Uint32 $>$ s)
\begin{DoxyCompactList}\small\item\em Constructor allows specificating the resource and segments. \end{DoxyCompactList}\item 
\mbox{\Hypertarget{classSimpleP2P_1_1FileRequest_a611bde99dbd62e155853e127fa102910}\label{classSimpleP2P_1_1FileRequest_a611bde99dbd62e155853e127fa102910}} 
std\+::vector$<$ Int8 $>$ \hyperlink{classSimpleP2P_1_1FileRequest_a611bde99dbd62e155853e127fa102910}{get\+\_\+resource\+\_\+header} () const
\begin{DoxyCompactList}\small\item\em Get the resource header. \end{DoxyCompactList}\item 
\mbox{\Hypertarget{classSimpleP2P_1_1FileRequest_ad3b33994ea628979673195fd4e3348af}\label{classSimpleP2P_1_1FileRequest_ad3b33994ea628979673195fd4e3348af}} 
std\+::vector$<$ Uint32 $>$ \hyperlink{classSimpleP2P_1_1FileRequest_ad3b33994ea628979673195fd4e3348af}{get\+\_\+segments} () const
\begin{DoxyCompactList}\small\item\em Get the segments\textquotesingle{} numbers. \end{DoxyCompactList}\end{DoxyCompactItemize}


\subsubsection{Detailed Description}
Carries info about a single file transfer request -\/ resource header and numbers of wanted segments. 

An instance of this class is created by T\+CP client, sent to T\+CP server, which passes it to the \hyperlink{classSimpleP2P_1_1FileManager}{File\+Manager} in order to get the requested segments of the requested file and send them to the T\+CP client. 

\subsubsection{Constructor \& Destructor Documentation}
\mbox{\Hypertarget{classSimpleP2P_1_1FileRequest_a6e50d4f7d07a46ddcd72745ebc6e9e81}\label{classSimpleP2P_1_1FileRequest_a6e50d4f7d07a46ddcd72745ebc6e9e81}} 
\index{Simple\+P2\+P\+::\+File\+Request@{Simple\+P2\+P\+::\+File\+Request}!File\+Request@{File\+Request}}
\index{File\+Request@{File\+Request}!Simple\+P2\+P\+::\+File\+Request@{Simple\+P2\+P\+::\+File\+Request}}
\paragraph{\texorpdfstring{File\+Request()}{FileRequest()}}
{\footnotesize\ttfamily Simple\+P2\+P\+::\+File\+Request\+::\+File\+Request (\begin{DoxyParamCaption}\item[{std\+::vector$<$ Int8 $>$}]{rh,  }\item[{std\+::initializer\+\_\+list$<$ Uint32 $>$}]{s }\end{DoxyParamCaption})}



Constructor allows specificating the resource and segments. 

The fields then can\textquotesingle{}t be modified, only get. 

The documentation for this class was generated from the following files\+:\begin{DoxyCompactItemize}
\item 
include/File\+Request.\+h\item 
src/File\+Request.\+cpp\end{DoxyCompactItemize}
 

\end{document}
